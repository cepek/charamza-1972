\selectlanguage{english}
\addcontentsline{toc}{section}{Abstract}

\chapter*{Abstract}
\section*{The Solution of Fundamental Problems of Least
          Squares Adjustment by Orthogonalization}

\begin{center}
František Charamza
\end{center}

\noindent
%
The monography focuses on the derivation and detailed
analyses of a numerically stable orthogonalization algorithm
(ORTON) for solving fundamental linear adjustment problems
with uncorrelated observations. This algorithm may be utilized
in the adjustment of intermediate and conditioned observations,
intermediate observations with conditions as well as
conditioned observations with unknown quantities for the purpose of
the determination of residuals, unknown quantities, functions
of adjusted values and reciprocal weights. ORTON is one of the
finite algorithms which do not require that normal equations
be formed. Apart from being highly numerically stable and
universal, the algorithm has other features helpful for practical
computations, for instance, diagnostic, enabling the
singularity of the problem to be identified and the inadequate loss of
significant digits to be specified, etc. The ORTON algorithm
may also be used for solving systems of linear algebraic
equations, not only those with square regular matrices, but also
those with singular or rectangular matrices respectively. It
may be applied in solving homogeneous systems as well.

Besides theoretical considerations, the monography also
provides detailed programming documentation of the ORTON 3
procedure, which defines the ORTON algorithm in the language ALGOL, as
well as ORTON 3 - ZKOUŠKA 1 programme, which illustrates the
possibility of using the ORTON 3 procedure in practise. Both
the procedure and the programme can be directly used for
calculations of adjustment problems on the ELLIOTT 503 computer and,
after some modifications, on other computers supplied with an
ALGOL compiler.
