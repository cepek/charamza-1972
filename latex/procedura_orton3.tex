\newcommand{\Parametr}[1]{
        \vspace{0.6ex}\underline{#1} }

Algoritmus \orig{112} ORTON, popsaný v kap. 6, realizuje procedura
ORTON3, sestavená v jazyce ALGOL pro počítač ELLIOTT 503.  Procedura
ORTON3 transformuje vstupní matici $A$ na matici $W$ podle schematu
%
\begin{align*}
  % https://tex.stackexchange.com/questions/639307/curly-braces-of-arbitrary-size
  \newcommand{\biggglB}[1]{\left\{\!\parbox{0pt}{\rule{0pt}{#1}}\right.}
  \tag{12.1}
  \vcenter{\hbox{
  \begin{tikzpicture}
    \draw (0,2) rectangle (1,3); \draw(0.5,2.5) node{$A_{11}$};
    \draw (1,2) rectangle (2,3); \draw(1.5,2.5) node{$A_{12}$};
    \draw (2,2) rectangle (3,3); \draw(2.5,2.5) node{$A_{13}$};
    \draw (0,1) rectangle (1,2); \draw(0.5,1.5) node{$A_{21}$};
    \draw (1,1) rectangle (2,2); \draw(1.5,1.5) node{$A_{22}$};
    \draw (2,1) rectangle (3,2); \draw(2.5,1.5) node{$A_{23}$};
    \draw (0,0) rectangle (1,1); \draw(0.5,0.5) node{$A_{31}$};
    \draw (1,0) rectangle (2,1); \draw(1.5,0.5) node{$A_{32}$};
    \draw (2,0) rectangle (3,1); \draw(2.5,0.5) node{$A_{33}$};
    %
    \draw(-.5,1.5) node{$A\; = $};
    \draw(3.4,2.5) node{\scriptsize $m_1$};
    \draw(3.4,1.5) node{\scriptsize $m_2$};
    \draw(3.4,0.5) node{\scriptsize $m_3$};
    %
    \draw(1,3.5)   node{$\overbrace{\hspace{1.8cm}}^{{n_1}}$};
    \draw(2.5,3.5) node{$\overbrace{\hspace{0.8cm}}^{{n_2}}$};%{}^{n_2}$};
    \draw(0.5,-0.4) node{${}^{m_2}$};
  \end{tikzpicture} }} \quad \longrightarrow \quad
  \mbox{ \scriptsize $m \;\; \biggglB{2.8cm} \;$}
  %
  \vcenter{\hbox{
    \begin{tikzpicture}
    \draw (0,2) rectangle (1,3); \draw(0.5,2.5) node{$W_{11}$};
    \draw (1,2) rectangle (2,3); \draw(1.5,2.5) node{$W_{12}$};
    \draw (2,2) rectangle (3,3); \draw(2.5,2.5) node{$W_{13}$};
    \draw (0,1) rectangle (1,2); \draw(0.5,1.5) node{$W_{21}$};
    \draw (1,1) rectangle (2,2); \draw(1.5,1.5) node{$W_{22}$};
    \draw (2,1) rectangle (3,2); \draw(2.5,1.5) node{$W_{23}$};
    \draw (0,0) rectangle (1,1); \draw(0.5,0.5) node{$W_{31}$};
    \draw (1,0) rectangle (2,1); \draw(1.5,0.5) node{$W_{32}$};
    \draw (2,0) rectangle (3,1); \draw(2.5,0.5) node{$W_{33}$};
    %
    \draw(1.5,3.4) node{$\overbrace{\hspace{2.8cm}}^{n}$};
    \draw(0.5,-0.4) node{\scriptsize \textcolor{white}{${}^{m_2}$}};
    \draw(3.6,1.5) node{$ = W\Punc{.}$};
  \end{tikzpicture} }}
\end{align*}
%
Předpokládá se, že
%
\begin{align*}
  \tag{12.2}
  \begin{aligned}
    &m_1 > 0, \qquad n_1 > 0, \qquad m_2 \ge 0,
    \qquad m_3 \ge 0, \qquad n_2 \ge 0,\\
    &m = m_1 + m_2 + m_2, \qquad n = n_1 + n_2\Punc{.}
  \end{aligned}
\end{align*}

\noindent
Je-li $m > 0$, můžeme procedurou řešit úlohy typu IC resp. CU,
formujeme-li vstupní matici $A$ např. podle (7.8) resp. (7.20),
Rozměr $m_2$ potom představuje počet podmínek mezi neznámými resp.
počet neměřených neznámých v podmínkových rovnicích. Je-li $m_2 = 0$,
potom jak jsme viděli v kap. 6, přechází ORTON v zobecněnou
ortogonalizaci matice
%
\begin{align*}
  \tag{12.3}
  \vcenter{\hbox{
    \begin{tikzpicture}
    \draw[thick] (0,1) rectangle (1,2); \draw(0.5,1.5) node{$A_{12}$};
    \draw (1,1) rectangle (2,2); \draw(1.5,1.5) node{$A_{13}$};
    \draw (0,0) rectangle (1,1); \draw(0.5,0.5) node{$A_{32}$};
    \draw (1,0) rectangle (2,1); \draw(1.5,0.5) node{$A_{33}$};
    \draw(2.4,1.5) node{\scriptsize $m_1$};
    \draw(2.4,0.5) node{\scriptsize $m_3$};
    %
    \draw(0.5,2.3) node{${}^{n_1}$};
    \draw(1.5,2.3) node{${}^{n_2}$};
    \draw(-.5,1.02) node{$A = $};
  \end{tikzpicture} }}
\end{align*}

\noindent
Formujeme-li matici $A$ ve (12.3) podle (3.19) resp. (3.37),
můžeme procedury užít k vyrovnání zprostředkujících resp.
podmínkových pozorování. Postupy, uvedenými v kap. 11, můžeme
procedurou ORTON3 rovněž hledat řešení soustav lineárních
algebraických rovnic a to nejen s regulárními, ale i singulárními,
případně obdélníkovými maticemi. Jedna z předností procedury
spočívá v možnosti její aplikace k řešení špatně podmíněných nebo
%
singulárních \orig{113} úloh. Procedura je koncipována tak, že
v případech, kdy řešení úlohy je numericky nejisté nebo není
jednoznačné, automaticky definuje a řeší náhradní jednoznačnou
úlohu v jistém smyslu blízkou výchozí úloze.

Při vlastní ortogonalizaci užívá ORTON3 modifikované
\name{GRAMOVY-SCHMIDTOVY} metody (MGS -- viz kap. 2) v uspořádání (2.11),
přičemž její výsledky koriguje doortogonalizací, do jisté míry
volitelnou. Při $m_2=0$ zpracovává ORTON3 sloupce matice $A$
volitelně buď v sekvenčním nebo selektivním uspořádání (kap. 9).
Při $m_2 \ne 0$ automaticky vychází ze selektivní varianty algoritmu;
takový postup je nutný, má-li být zajištěna nezbytná regularita
matice $A_{21}$ (kap. 6).

Dříve než je kterýkoliv sloupec a matice $A$ podroben
ortogonalizaci, normalizuje se dělením absolutně největším prvkem ze
svých prvních $(m_1 + m_2)$ prvků
%
$ { \max_{1 \le i \le m_1+m_2} }| a_i | = \Vert a \Vert_\infty $
%
podle schematu
%
\begin{align*}
  \tag{12.4}
  a / \Vert a \Vert_\infty \rightarrow a \Punc{.}
\end{align*}
%
Účelem této vstupní normalizace je odstranit nebezpečí z
přetečení resp. podtečení při extrémně velkých nebo malých řádových
velikostech prvků matice $A$ a získat, jak jsme viděli v kap. 9,
měřítko pro posouzení numerické stability výpočtu. Pro přesnost
je třeba dodat, že v některých výjimečných případech se
bezprostředně nedělí normou
%
$ \Vert a \Vert_\infty\,,$
%
ale veličinou
%
$ \Vert a \Vert_\infty \cdot \mathrm{magn}$
%
nebo
%
$ \Vert a \Vert_\infty / \mathrm{magn}$,
%
kde $\mathrm{magn}$ je jeden z parametrů procedury.






Popis procedury ORTON3 obsahuje příl. A. Příklad jejího
praktického užití uvedeme v kap.~13. Pro vytvoření představy o
rychlosti, s jakou je procedurou ORTON3 na počítači ELLIOTT 503
zpracována matice $A$, uveďme empirickou formuli
%
$\mathrm{T}^{\min} \approx 2.10^{-5}m_1n_1^2$,
%
udávající dobu potřebnou pro ortogonalizaci základní submatice
$A_{12}$ v (12.3) sekvenční variantou algoritmu. Formule platí za
předpokladu, že nebyla aplikována doortogonalizace.


\section{Parametry procedury}

Procedura \orig{114}ORTON3 má 12 parametrů s vlastnostmi vyjádřenými
hlavičkou procedury

\begin{lstlisting}
procedure ORTON3(A,ml,m2,m3,nl,n2,defect, tol,tolmin,  @:\Tag(12.5):@
                 magn,change,INDEX);
value ml,m2,n2,defect,tol,tolmin,magn,change;
integer ml,m2,mj,nl,n2,defect; real tol,tolmin,magn;
boolean change; array A; integer array INDEX;
\end{lstlisting}

\noindent
Popíšeme význam jednotlivých parametrů.

\Parametr{Parametr \TT{A}} je identifikátorem dvourozměrného pole, v němž
musí být při vstupu do procedury uložena zpracovávaná matice
$A$ (12.1). Předpokládá se, že dolní meze obou indexů jsou
rovny jedné. Při výstupu z procedury bude v poli \TT{A} uložena matice
$W$, získaná podle (12.1) aplikací algoritmu ORTON na matici $A$.
Je-li $m_2 = 0$, zaujímá matice $W$ týž paměťový prostor s~mezemi
$[1:m,1:n]$, jako matice $A$. Při $m_2 \ne 0$, mohou být paměťové nároky
matice $W$ větší než nároky výchozí matice $A$. Stane se tak při
řešení singulárních úloh (viz parametr \TT{tolmin}) tehdy, bude-li
hodnost $h$ matice $[A_{21} A_{22} ]$ menší než $m_2$, (singularita
druhéno druhu). Jak jsme viděli v kap. 8, rozšiřuje v takových
případech procedura ORTON3 výchozí matici $A$ generováním $(m_2 - h)$
nových sloupců a řádků, takže požadovaný prostor ohraničují
meze indexů
%
\begin{align*}
  \tag{12.6}  [ 1:m+m_2 - h, 1:n + m_2 - h]\Punc{.}
\end{align*}
%
Je zřejmé, že meze (12.6) musí být respektovány při deklaraci
skutečného parametru (pole), odpovídajícího formálnímu parametru
\TT{A}. Neznáme-li hodnost $h$, může být užito \uv{volnější}
de\-kla\-ra\-ce např.
%
\begin{lstlisting}
   array XXX [1:m+defect, 1:n+defect],   @:\Tag(12.7):@
\end{lstlisting}
%
kde hodnota proměnné \uv{\TT{defect}} není menší než rozdíl $(m_2-h)$.


\Parametr{Parametry \TT{m1}, \TT{m2}, \TT{m3}, \TT{n1}, \TT{n2}}
určují v souladu s (12.1) strukturu zpracovávané matice. Hodnoty
\orig{115} proměnných \TT{m1}, \TT{m2} a \TT{n2} se voláním procedury ORTON3
nemění. Hodnoty proměnných \TT{m3} a \TT{n1} se nemění pouze tehdy,
pokud matice \TT{A} nebyla rozšířena generováním nových sloupců a
řádků. Byla-li matice \TT{A} rozšířena o $(m_2-h)$ sloupců a řádků (viz
popis parametru \TT{A}), pak o stejnou veličinu $(m_2-h)$ budou
zvětšeny hodnoty proměnných \TT{m3} a \TT{nl}. Při výstupu z procedury
ORTON3 je tedy hodnotami parametrů \TT{m1}, \TT{m2}, \TT{m3}, \TT{n1}
a \TT{n2} ve všech případech vyjádřena struktura výsledné matice \TT{W}.

\Parametr{Parametr \TT{defect}} se uplatňuje při řešení singulárních úloh
druhého druhu, uvedených při popisu parametru \TT{A} $(m_2 \ne 0, h<m_2)$.
Stejně jako ve (12.7) musí být v oněch případech při vyvolání
procedury ORTON3 hodnota proměnné defect větší nebo rovná
rozdílu mezí dimenzí $m_2$, a hodností $h$ matice $[A_{21} A_{22}]$ Je-li
$m_2=0$ nebo současně $m_2 \ne 0$ a $h=m_2$, pak je vhodné, aby hodnota
odpovídajícího skutečného parametru byla rovna nule.

\Parametr{Parametry \TT{tol}, \TT{tolmin}, \TT{magn}} představují v
podstatě kritéria, jimiž je řízeno použití doortogonalizace a podle
nichž je posuzována singularita úlohy. Probereme nejprve jednočušší
případ $m_2 = 0$.



Je-li $m_2=0$, redukuje se algoritmus ORTON na zobecněnou
ortogonalizaci (12.3). Přitom je pro přesnost řešení zřejmě
rozhodující především jistota, s jakou se podaří ortogonalizovat
sloupce základní submatice $A_{12} \rightarrow W_{12}$.  Označme $a$
libovolný sloupec základní submatice, $\widetilde w$ sloupec vzniklý
ortogonalizací a např.  podle (2.1,) a $w$ sloupec, který vznikl
normalizací $\widetilde W$.  V~kap. 9 jsme viděli, že chyba vektoru
$w$ může být měřena velikostí normy $d = \Vert \widetilde W \Vert$
přibližně v tom smyslu, že logaritmus této normy aproximuje počet
cifer, ztřacených při výpočtu $w$ (selektivní varianta
algoritmu). Je-li např.  $\mathcal{O}(\Vert \widetilde W \Vert) =
10^2$ 1ze očekávat ztrátu přibližně až dvou platných cifer při
zpracování $a$. Aby bylo možno zabránit velkým ztrátám přesnosti, jsou
sloupce, při jejichž ortogonalizaci ztráta přesnosti hrozí,
zpracovávány procedurou ORTON3 individuálním způsobem. Postup
zpracování závisí na hodnotách parametrů (tolerancí) \TT{tol} a
\TT{tolmin}, jak ukazuje tab. 12.1. \orig{116}

%
% --------------------------------------------------------------------
%
\begin{table}[!htb]
\centering\setlength{\extrarowheight}{2pt}
\begin{tabular}{|p{0.25\linewidth}|p{0.695\linewidth}|}
\hline
 \multicolumn{1}{|c|}{Velikost $\Vert \widetilde w \Vert$} &
 {Způsob zpracování sloupce v základní submatici}\\
\hline
\centering $\Vert \widetilde w \Vert \ge $ \TT{tol} &
Ortogonalizace bez užití doortogonalizace \\
\hline
\centering\TT{tolmin} $\le \Vert \widetilde w \Vert < $ \TT{tol} &
Ortogonalizace korigovaná doortogonalizací.
Při zpracování sloupců selektivním způsobem
proběhne doortogonalizace až tehdy, nebudou-li
existovat sloupce základní submatice, které
vedou k větší hodnotě normy $\Vert\widetilde w\Vert$.
U sekvenční varianty algoritmu následuje
doortogonalizacebezprostředně po ortogonalizaci.
\\
\hline
\centering $\Vert\widetilde w \Vert < $ \TT{tolmin}&
Singularita prvního druhu. Sloupec bude vyloučen
jako lineárně závislý.
\\
\hline
\end{tabular}\\[1.5em]
{Tab. 12.1}
\end{table}
%
% --------------------------------------------------------------------
%

\noindent
Parametru \TT{magn} se při $m_2=0$ neužívá. Hodnota odpovídajícího
skutečného parametru může být libovolná.


Je-li $m_2 \ne 0$, proběhnou obecně všechny čtyři kroky algoritmu
ORTON. Parametry \TT{tol}, \TT{tolmin} a \TT{magn} mají pak poněkud širší
význam než tomu bylo ve zvláštním případě $m_2=0$. Užívají se totiž
mj.~k~rozlišení singularit dvou různých typů (\TT{tolmin}) a
zabezpečují vhodnou selekci sloupců při ortogonalizaci (\TT{tol}).
Způsob řízení algoritmu tolerancemi \TT{tol} a \TT{tolmin} popisuje tab. 12.2.
Pro jednoduchost je uvažováno pouze zpracování sloupců submatice

\begin{align*}
  \tag{12.8}
  \vcenter{\hbox{
    \begin{tikzpicture}
    \draw (0,1) rectangle (1,2); \draw(0.5,1.5) node{$A_{11}$};
    \draw (1,1) rectangle (2,2); \draw(1.5,1.5) node{$A_{12}$};
    \draw (0,0) rectangle (1,1); \draw(0.5,0.5) node{$A_{21}$};
    \draw (1,0) rectangle (2,1); \draw(1.5,0.5) node{$A_{22}$};
  \end{tikzpicture} }} \Punc{.}
\end{align*}
%
\orig{117}
%
% --------------------------------------------------------------------
%
\begin{table}[!htb]
{\centering\setlength{\extrarowheight}{2pt}
\begin{tabular}{|c|p{0.325\linewidth}|p{0.325\linewidth}|}
\hline
 \multirow{2}{*}{Velikost $\Vert \widetilde w \Vert^{\star}$}
 &
 \multicolumn{2}{p{0.7\linewidth}|}
 {Způsob zpracování sloupců submatice (12.8)}
 \\
 \cline{2-3}
 &
 \multicolumn{1}{p{0.325\linewidth}|}
 { Výběr a zpracování prvních $m_2$, sloupců
   zajišťujících regularitu $A_{21}$
 }
 & \multicolumn{1}{p{0.325\linewidth}|}
 {Zpracování zbývajících sloup\-ců
 } \\
\hline %----------------------------------------------------
\centering $\Vert \widetilde w \Vert \ge $ \TT{tol} &
Ortogonalizace bez doortogonalizace
&
Ortogonalizace korigovaná do\-ortogonalizací
\\
\hline  %----------------------------------------------------
\centering\TT{tolmin} $\le \Vert \widetilde w \Vert < $ \TT{tol} &
\multicolumn{2}{p{0.6\linewidth}|}
            {Ortogonalizace korigovaná
              doortogonalizací.}
\\
\hline
\centering $\Vert\widetilde w \Vert < $ \TT{tolmin}
&
%
Singularita druhého druhu. Bude generován druhu nový sloupec a řádek
matice $A$.
%
&
%
Singularita prvního druhu. Sloupec bude vyloučen jako lineárně
závislý.
%
\\
\hline
\end{tabular}\\[0.9em]
}

${}^\star$ {\small
   Při výpočtu prvních $m_2$ sloupců matice $W$ je
   norma $\Vert \widetilde w \Vert$
   počítána z prvků odpovídajících spodní submatici ve (12.8).
   ~Při výpočtu ostatních sloupců naopak z prvků odpovídajících
   horní submatici (viz definici algoritmu ORTON).}
\begin{center}\vspace{-1.0em}
         Tab 12.2
\end{center}
\end{table}
%
% --------------------------------------------------------------------
%

\noindent
Doortogonalizace při $\Vert\widetilde w\Vert < \TT{tol}$ stejně
jako generování slourců a řádků, příp. vylučování sloupců proběhnou až
tehdy, nebude-li existovat sloupec vedoucí k větší hodnotě
normy $\Vert\widetilde w\Vert$.

Užití norem $\Vert\widetilde w\Vert$ jako měřítek přesnosti
ortogonalizovaných vektorů, stejně jako stanovení tolerancí \TT{tol}
a \TT{tolmin}, vychází z předpokladu, že délka výchozího vektoru $a$
předběžně normalizovaného podle (12.4) je přibližně jednotková
(kap.9). Tento předpoklad je splněn při $m_2=0$. Je-li $m_2\ne0$
a prvky v horní části submatice (12.8) budou řádově jiné než prvky
v její dolní
části, pak splněn být nemusí\footnote
{
Vektor $a$, kterému odpovídá norma $\Vert\widetilde w\Vert$,
je totiž tvořen prvky, ležícími buď jen v horní nebo jen spodní
submatici ve (12.8), zatímco normalizační faktor ve (12.4)
se vybírá ze všech prvků příslušného sloupce ve (12.8).
}.
%
Je proto v proceduře ORTON3 zaveden
%
\orig{118} parametr
\TT{magn}, řádově udávající poměr velikosti prvků v horní
a dolní části submatice (12.8). Je-li např. $10^m$ resp. $10^n$ řád
prvků v horní resp. dolní části (12.8), pak hodnota skutečného
parametru, odpovídajícího parametru \TT{magn} bude $10^{m-n}$.  Při
stanovení \TT{magn} se uvažují pouze sloupce, mající v horní i dolní
části nenulové prvky. Sloupce, mající nenulové prvky jen v horní nebo
jen v dolní části se v úvahu neberou. Procedura ORTON3 respektuje
hodnotu parametru \TT{magn} tím, že při $\TT{magn}\ne1$ interně
modifikuje zadané hodnoty tolerancí \TT{tol}
a \TT{tolmin}. Modifikované tolerance procedura tiskne. Zavedením
parametru \TT{magn} se způsob stanovení výchozích hodnot
tolerancí \TT{tol} a \TT{tolmin} nemění.  Stejně jako předtím jsou
tolerancemi definovány jisté kritické ztráty platných cifer, sloužící
k řízení výpočtu podle tab. 12.1 a 12.2.


Jak při $m_2=0$, tak při $m_2\ne0$ se předpokládá, že
%
\begin{align*}
   \tag{12.9}  0 < \TT{tolmin} \le \TT{tol}\Punc{.}
\end{align*}
%
Numerické hodnoty obou tolerancí závisí na charakteru řešené
úlohy. Pro hrubou orientaci uveďme, že při užití procedury k řešení
menších vyrovnávacích úloh se osvědčily např. hodnoty
$\TT{tol}=10^{-1}$ a $\TT{tolmin}=10^{-6}$ (bez ohledu na
velikost \TT{magn}).



\Parametr{Parametr \TT{change}}
umožňuje volbu mezi sekvenčním a selektivním
způsobem zpracování sloupců matice $A$. Je-li hodnota skutečného
parametru $\mathbf{true}$, pak ORTON3 užije selektivní varianty
algoritmu.  Je-li
jeho hodnota $\mathbf{false}$, pak užije varianty sekvenční,
ovšem pouze tehdy,
bude-li současně $m_2=0$. V případech, že je současně
$\TT{change} \equiv \mathbf{false}$ a
$m_2\ne0$, bude hodnota parametru \TT{change} interně změněna na
$\mathbf{true}$ a bude
užito selektivní varianty algoritmu.


\Parametr{Parametr \TT{INDEX}} je identifikátorem jednorozměrného
celočíselného pole s mezemi $[\;1:n+(m_2-h)\;]$, kde $(m_2-h)$ je počet
sloupců generovaných procedurou ORTON3 v případě singularity druhého
druhu. Stejně jako u parametru \TT{A} může být i zde při neznalosti
hodnosti $h$ užito deklarace skutečného parametru ve tvaru
%
\begin{lstlisting}
   integer array XXX [1:m+defect, 1:n+defect],   @:\Tag(12.10):@
\end{lstlisting}
%
kde \orig{119} hodnota proměnné \uv{\TT{defect}} není menší než
$(m_2-h)$.  Při výstupu z procedury ORTON3 udává \TT{INDEX}
posloupnost, v jaké byly zpracovány sloupce matice
$A$. Proměnná \TT{INDEX[i]} obsahuje totiž index sloupce matice $A$,
jehož ortogonalizací vznikl i-tý sloupec výsledné matice $W$
$(i=1,2,\ldots,n)$. Sloupcům, které byly generovány procedurou, je
přiřazen záporný index. Např. první generovaný sloupec má index rovný
$-(n_1+1)$.



\section{Význam tisků}

Při vyvolání procedury ORTON3 dochází k tisku
diagnostických informací, podrobně popisujících průběh transformace
%
$A \rightarrow W$.
%
Pomocí nich je možno analyzovat singulární úlohy, posuzovat
přesnost řešení, sledovat výběr sloupců při užití selektivní
varianty algoritmu apod. Tištěné informace jsou provázeny
objasňujícími texty. V následujícím přehledu vysvětlíme význam
jednotlivých tištěných hodnot, přičemž budeme znakem \TT{X}
symbolizovat tisk celých čísel a znakem \TT{X$_{10}$X} tisk čísel v
semilogaritmickém tvaru. Pro větší přesnost popisu budeme rozlišovat
indexy sloupců matice $A$ a indexy sloupců pole \TT{A}. Takové
rozlišení je nutné u selektivní varianty algoritmu, kde může v poli
\TT{A} dojít k explicitním přesunům sloupců výchozí matice $A$.


\newenvironment{Minipage}{%
\noindent\begin{minipage}{\linewidth}
\vspace{2.2ex}
}{\end{minipage}}

{\begin{Minipage}
\TT{TOL~~~ = X$_{10}$X}\\
\TT{TOLMIN = X$_{10}$X}\\
\end{Minipage}
%
\noindent
Tisk tolerancí \TT{tol} a \TT{tolmin} (viz parametry \TT{tol},
\TT{tolmin}, \TT{magn}),
s nimiž procedura pracuje. Tyto hodnoty jsou při $m_2=0$ rovny
hodnotám odpovídajících skutečných parametrů. Je-li $m_2\ne0$, tisknou
se v závislosti na hodnotě \TT{magn} tolerance interně
modifikované. Úprava tolerancí je obecně jiná při určování prvních
$m_2$ sloupců $w_i$ matice $W$ než při určování zbývajících sloupců.
Vlastní modifikace tolerance \TT{tol} probíhá podle tab. 12.3.
\orig{120}

\begin{table}[!htb]
\centering
\begin{tabular}{|c|c|c|}
\hline
\diagbox{\TT{magn}}{\TT{i~~~}} & $\le m_2$ & $> m_2$ \\
\hline%\hline
  $ \le 1$ & \TT{tol} & \TT{tol} $\times$ \TT{magn} \\
\hline
  $ >1 $   & \TT{tol/magn} & \TT{tol} \\
\hline
\end{tabular}\\[1.5ex]
Tab. 12.3
\end{table}

\noindent
Tolerance tolmin je modifikována analogicky. Z tab. 12.3 je
zřejmé, že při $\TT{magn}=1$ se vstupní hodnoty tolerancí nemění.


\begin{Minipage}
\TT{COLUMN NO.: X}\\
\end{Minipage}

\noindent Záhlaví odstavce výstupního textu, popisujícího postup při
vytváření X-tého sloupce matice~$W$.

\begin{Minipage}
\TT{TRY COL. NO. X}\\
\end{Minipage}

\noindent Tiskne se index toho sloupce pole \TT{A},
který bude podroben ortogonalizaci. Při užití sekvenční varianty je to
současně index sloupce matice $A$. U selektivní varianty může navíc
dojít k tisku informace

\begin{Minipage}
\TT{SMALL NORM}\\
\end{Minipage}

\noindent
označující, že již v některé předcházející etapě výpočtu byl učiněn
neúspěšný pokus o ortogonalizaci X-tého sloupce. Tím míníme, že dříve
započaté zpracování dotyčného sloupce nemohlo být dokončeno pro příliš
malou hodnotu normy $\Vert\widetilde w\Vert$ (viz popis
parametrů \TT{tol} a \TT{tolmin}). Takového zjištění užívá procedura
při řízení dalšího postupu.

\begin{Minipage}
\TT{ABSMAX [X,X] = X$_{10}$X}\\
\end{Minipage}

\noindent
Tisk absolutní hodnoty absolutně největšího prvku z prvních
$(m_1 + m_2)$ prvků ortogonalizovaného sloupce. V hranatých
závorkách se tisknou indexy tohoto prvku, charakterizující jeho
umístění v poli \TT{A}. První index je řádkový, druhý sloupcový.
Navíc dojde k tisku informace \orig{121}

\begin{Minipage}
\TT{ZERO COLUMN}\\
\end{Minipage}

\noindent
v těch případech, kdy \TT{ABSMAX [X,X] = O}.


\begin{Minipage}
\TT{NORM [X] = X$_{10}$X}\\
\end{Minipage}
%
Tiskne se hodnota normy $\Vert\widetilde w\Vert$, k níž vedla
ortogonalizace a případně doortogonalizace sloupce, uvedeného textem
\uv{\TT{TRY~COL.~NO.}} (viz popis parametrů \TT{tol} a \TT{tolmin})
nebo textem \uv{\TT{GENERATED COL. NO.}}.


\begin{Minipage}
\TT{< T0L}\\
\TT{< TOLMIN}\\
\end{Minipage}
%
Informace o tom, že norma $\Vert\widetilde w\Vert$ vektoru,
který lze identifikovat z kontextu, je menší než tolerance
\TT{tol} resp. \TT{tolmin}.


\begin{Minipage}
\TT{REPEATED ORTHOG.}\\
\end{Minipage}
%
Bude aplikována doortogonalizace.


\begin{Minipage}
\TT{LIN. DEP. COL. NO. X}\\
\end{Minipage}
%
Dojde-li k tomuto tisku při zpracování prvních n sloupců
matice $A$, je jím signalizována singularita prvního druhu (kap.8),
Stane-li se tak při zpracování posledních $n_2$ sloupců matice $A$,
pak byl v matici
%
$\left[ \begin{array}{c}A_{13}\\A_{23}\end{array} \right]$
%
nalezen sloupec, který je lineární kombinací sloupců matice
%
$ \left[
\begin{array}{cc}
A_{11} & A_{12} \\ A_{21} & A_{22}
\end{array}
\right]{.} $
%
~\TT{X} udává index \uv{singulárního}
sloupce výchozí matice $A$ a to i tehdy, když u selektivní varianty
došlo k záměnám sloupců. Pokud byla singuliarita zjištěna u některého
z prvních $n_1$ sloupců matice $A$, bude následovat tisk posledních
$m_3$ prvků odpovídajícího vektoru $\Vert\widetilde w\Vert$ Jak jsme
viděli v kap. 8, lze těchto hodnot užít např. při posuzování
řešitelnosti podmínkových rovnic, při hledání koeficientů pro
vyjádření jedné podmínkové rovnice jako lineární kombinace jiných
apod.

\orig{122}
\begin{Minipage}
\TT{PARENT COL. NO. X}\\
\end{Minipage}
%
Index sloupce výchozí metice $A$, jehož ortogonalizací vznikl
sloupec matice $W$ uvedený textem \uv{\TT{COLUMN NO.}}. Indexy sloupců,
generovaných uvnitř procedury, jsou záporné.


\begin{Minipage}
\TT{REPLACE COL. NO. X}\\
\end{Minipage}
%
Index sloupce výchozí matice $A$, jehož ortogonalizací vznikl sloupec
matice $W$ uvedený textem \uv{TT{COLUMN NO.}}. Indexy sloupců,
generovaných uvnitř procedury, jsou záporné.

\begin{Minipage}
\TT{REPLACE X $\to$ X $\to$ X $\to$ X}\\
\end{Minipage}
%
Informace o záměnách sloupců v poli \TT{A} u selektivní varianty
algoritmu. Mechanismus užívaný při záměnách vyložíme pro ilustraci na
konkrétním příkladě. Předpokládejme, že. např. první dva sloupce $a_1,
a_2$ matice $A$ byly transformovány na první dva sloupce $w_, w_2$
matice $W$. Při hledání třetího sloupce matice $W$ nechť se ukázalo,
že sloupce $a_3$ a $a_4$, vedou k normám $\Vert\widetilde w_3\Vert$ a
$\Vert\widetilde w_4\Vert$ menším, než tolerance \TT{tol}, sloupec
$a_5$ k normě $\Vert\widetilde w_3\Vert$ větší než
\TT{tol}. Je-li v dané úloze např. $n_1=10$, dojde potom k přesunům
podle schematu {\TT{REPLACE 3 $\to$ 10 $\to$ 5 $\to$ 3}.
Z příkladu je patrná
zásada, podle níž jsou kosoúhlé sloupce přesunovány co nejdále
vpravo od sloupce matice $W$, který je právě určován. Je-li
takových sloupců více, vytváří se při pravém okraji odpovídající
části pole \TT{A} skupina \uv{málo kvalitních} sloupců, které označujeme
jako


\begin{Minipage}
\TT{RUBBISH .}\\
\end{Minipage}
%
Dojde-li k tisku tohoto slova, pak již nelze najít žádný sloupec,
který by vedl k normě $\Vert\widetilde w\Vert \ge tol$. V takovém
případě může být tištěna informace

\begin{Minipage}
\TT{UPDATE COL. NO. X}\\
\end{Minipage}
%
spojená s ortogonalizací X-tého sloupce pole \TT{A} k těm sloupcům
matice $W$, k nimž dosud nebyl ortogonalizován. Takovým způsobem se
upravují všechny \uv{málo kvalitní} sloupce ze skupiny \TT{RUBBISH.}


\begin{Minipage}
\TT{BEST COL. NO. X WITH NORM X$_{10}$X}\\
\end{Minipage}
%
Index \TT{X} sloupce (v poli \TT{A}) ze skupiny \TT{RUBBISH}, který
vede k maximální hodnotě \TT{X$_{10}$X} normy $\Vert\widetilde
w\Vert$.\orig{123}


\begin{Minipage}
\TT{REPLACE X $\to$ X}\\
\end{Minipage}
%
Informace o přesunu sloupců při zjištění singularity druhého
druhu. Cílem přesunu je uvolnit potřebný prostor v poli \TT{A},
dříve než bude generován nový sloupec. Např. před generováním
prvního sloupce bude tištěno schema
\uv{\TT{REPLACE (n${}_1$+1) $\to$ (n+1)}}.


\begin{Minipage}
\TT{GENERATED COL. NO. X}\\
\end{Minipage}
%
Index \TT{X} sloupce v poli \TT{A}, který byl procedurou generován při
zjištění singularity druhého druhu. Následuje tisk prvků
generovaného sloupce s indexy
%
$m_1+1,m_1+2,\ldots,m_1+m_2$/
%
Hodnoty tištěných
prvků získává ORTON3 voláním lokálně deklarovaného generátoru
náhodných čísel \uv{\TT{random}} [45], [46], [55]. Parametry procedury
\TT{random} jsou přitom voleny tak, aby byla vytvářena náhodná čísla,
rovnoměrně rozdělená v intervalu <-1,1>.


\begin{Minipage}
\TT{GENERATED COL. IS LIN. DEP.}\\
\end{Minipage}
%
Tato informace se tiskne tehdy, vede-li generovaný sloupec k
hodnotě normy $\Vert\widetilde w\Vert< \TT{tol}$.

Pro ilustraci uvedeme na závěr ukázku tisku diagnostických
informací, získaných při řešení konkrétní úlohy s maticí
%
\begin{align*}
\tag{12.11}
A =
\begin{array}{|rr|r|r|}
\hline
 ~~1 &  ~~2 &  ~~1 &  ~~0 \\
 2 &  3 &  2 &  1 \\
 3 &  4 &  3 &  0 \\
 4 &  5 &  4 &  0 \\
 5 &  6 &  3 &  0 \\
 6 &  7 &  2 &  0 \\
 7 &  8 &  1 &  0 \\
 \hline
 1 &  2 &  1 &  1 \\
 3 &  6 &  3 &  0 \\
 \hline
 4 & -7 &  9 &  8 \\
 \hline
\end{array}\Punc{.}
\end{align*}

\noindent
Submatice $[A_{21} A_{22}]$ má v našem příkladě hodřost jedna, bude
tedy při vyvolání procedury ORTON3 generován v matici $A$ nový sloupec
(a řádek). Volíme-li hodnoty tolerancí $\TT{tol} = 10^{-1}$ a
$\TT{tolmin} = 10^{-6}$, pak při $\TT{magn}=1$ dostaneme tisk ve tvaru
\orig{124}

\newcommand{\X}[2]{#1${}_{\texttt{10}}$#2\!\!}% jina verze v reseni_soustav.tex
\newpage
%\enlargethispage{4\baselineskip}
{\small%
\begin{Verbatim}[commandchars=\\\{\},codes={\catcode`$=3\catcode`_=8}]
TOL   = \X{1.0}{-01}
TOLMIN= \X{1.0}{-06}

COLUMN NO.:   1
TRY COL. NO.   1
ABSMAX[   7,   1]=\X{6.99999998}{+00}
NORM[   1]= \X{4.51753951}{-01}

COLUMN NO.:   2
TRY COL. NO.   2
ABSMAX[   7,   2]= \X{6.99999999}{+00}
NORM[   2]= \X{2.08250059}{-09} <TOL
TRY COL. NO.   3
ABSMAX[   4,   3]= \X{3.99999999}{+00}
NORM[   3]= \X{2.08250059}{-09} <TOL
RUBBISH
BEST COL. NO.   2 WITH NORM \X{2.08250059}{-09} <TOLMIN
REPLACE   4  -->   5
GENERATED COL. NO.   4
   8   \X{6.65240091}{-02}
   9   \X{3.32670026}{-01}
NORM[   4]= \X{4.20797978}{-02}  <TOL
GENERATED COL. ID LIN. DEP.
   8  \X{-3.36649847}{-01}
   9   \X{3.12750760}{-01}
NORM[   4]= \X{4.19539473}{-01}
REPLACE   2  -->   4   -->   4   -->   2
PARENT COL. NO.  -4

TOL   = \X{1.0}{-01}
TOLMIN= \X{1.0}{-06}

COLUMN NO.:   3
TRY COL. NO.   3
NORM[   3]= \X{1.87082869}{+00}
REPEATED ORTHOG.
NORM[   3]= \X{1.87082869}{+00}

COLUMN NO.:   4                   COLUMN NO.:   5
TRY COL. NO.   4                  TRY COL. NO.   5
NORM[   4]= \X{5.28221412}{-01}        ABSMAX[   2,   5]= \X{1.00000000}{+00}
REPEATED ORTHOG.                  NORM[   5]=\X{8.62736276}{-01}
NORM[   4]= \X{5.28221412}{-01}        REPEATED ORTHOG.
PARENT COL. NO.   2               NORM[   5]= \X{6.62736276}{-01}
                                  PARENT COL. NO.   4
\end{Verbatim}
}

\noindent
Tištěné údaje nevyžadují zvláštního komentáře. Připomeneme jen,
že v odstavcích \TT{COLUMN NO.: 1} a \TT{COLUMN NO.: 2} se všechny normy
počítají z prvků odpovídajících 8. a 9. řádku v poli \TT{A}. Ve
zbývajících odstavcích odpovídají normy prvkům v prvních sedmi řádcích
pole \TT{A}.
