\selectlanguage{russian}
\addcontentsline{toc}{section}{Резюме}

\chapter*{Резюме}
\section*{Решение основных задач
          уравнительного вычисления\\
          методом ортогонализации}
\begin{center}
          Франтишек Харамза
\end{center}

\noindent
В монографии выведен и подробно исследован численно
устойчивый ортогонализационный алгоритм (ORTON) для решения
основных линейных задач уравнительного вычисления при
некоррелированных наблюдениях. Алгоритм может быть использован при
уравнивании косвенных и условных измерений, при уравнивании
косвенных измерений, связанных условиями и при уравнивании
условных измерений с неизвестными, для вычисления поправок,
неизвестных, функций уравненных величин и весовых коэффициентов,
ORTON относится к числу финитных алгоритмов, не требующих
составления нормальных уравнений. Для вычислительной практики кроме
высокой численной устойчитэсти алгоритма ценны также его
некоторые дальнейшие свойства, напр, диагностические, дающие
возможность установить вырожденность задачи, указать на
чрезмерную потерю значащих цифр и т.п, Алгоритм ORTON может быть
использован также для решения систем линейных алгебраических
уравнений, причем не только с регулярными ква дратными
матрицами, но и с матрицами вырожденными или прямоугольными.
Алгоритм пригоден и для решения однородных систем.

Кроме теоретических рассуждений в работе приведена
подробная документация процедуры ORTON 3, являющейся записью
алгоритма ORTON на языке ALGOL, и программы ORTON 3 - ZKOUŠKA 1,
которая показывает возможности практического применения
процедуры ORTON 3. Процедура и программа могут выть
непосредственно использованы для решения уравнительных задач
на машине ELLIOTT 503 а после незначительных изменений и на других
машинах, снабженных транслятором с языка ALGOL.
