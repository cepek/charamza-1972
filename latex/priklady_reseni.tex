{ \newcommand{\TS}[1]{\textsuperscript{#1}}

Uvedeme \orig{90} numerické příklady ortogonalizace dvou mátic,
odpovídajících špatně podmíněným vyrovnávacím úlohám. Hlavní
pozornost přitom věnujeme odhadům chyb vektorů $w_i$ podle vzorců z
kap. 9 a jejich porovnání se skutečně dosaženými chybami.


\Subsubsection{A} V prvním příkladě budeme pracovat s maticí $A(11 \times 3)$,
jejíž sloupce získáme v souladu s~(9.12) podle předpisu
%
\begin{align*}
  \tag{10.1}
  a_1 = b_1, \quad a_2 = b_1 + db_2, \quad a_3 = b_2 + db_3,
\end{align*}
%
kde $b_1$, $b_2$ a $b_3$ tvoří ortonormální soustavu a d=0,01.%
\footnote{
Lze dokázat, že pro takto definivanou matici $A$ platí
$\cond (A^TA) \approx 4.10^8$. Pokud by tedy matice $A$ byla např.
maticí soustavy rovnic oprav, bude odpovídající vyrovnávací
úloha velmi špatně podmíněná.
}
%
Abychom mohli posoudit skutečnou chybu vektorů $w_i$ získaných při
numerickém výpočtu, najdeme vektory $w_i$ nejprve obecně. Přitom budeme
uvažovat obě varianty ortogonalizačního algoritmu.

Sekvenční varianta:~~ Podle (9.13) vede ortogonalizace vektorů
(10.1) k vektorům $w_i = b_i$ $(i=1,2,3)$.


Selektivní varianta:~~ Z úvah o vzájemné poloze vektorů $a_i$
(tab. 9.1) bezprostředně vyplývá, že při užití selektivní varianty
budou tyto vektory ortogonalizovány v pořadí $a_1,a_3,a_2$. Přitom
dostaneme ovšem vektory $w_i$ obecně jiné než u sekvenční varianty.


Označíme-li $\widetilde w_i$  $(i=1,2,3)$ nenormalizované vektory získané
ortogonalizací vektorů $a_j$ v~pořadí $a_1,a_3,a_2$ a abstrahujeme-li od
úvodního dělení vektoru $a_i$ jeho některou normou, bude
%
\begin{align*}
  \tag{10.2}
  \widetilde w_1 &= a_1 = b_1,    &w_1 &= b_1\\
  \tag{10.3}
  \widetilde w_2 &= a_3 - (a_3,w_1)w_1 = a_3,
  &w_2 &= \widetilde w_2 / \Vert \widetilde w_2 \Vert =
  (1+d^2)^{-{1\over2}}a_3 \approx (1-0.5d^2)a_3\\
  \intertext{a analogicky}
  \tag{10.4}
  \widetilde w_3 &= -d^2b_3 + d^3a_3,
  &w_3 &\approx (1+0,5d^2)(-b_3+da_3).
  \\
\end{align*}


Ukážeme \orig{91} ještě, jakým způsobem lze najít soustavu výchozích
ortonormálnícn vektorů $b_i$. Prvky každého vektoru $b_i$ můžeme např.
považovat za funkční hodnoty polynomů ortonormálních na určité množině
bodů. Tyto hodnoty jsou pro ekvidistantně rozložené body tabelovány
např. v [54]. V naší úloze pracujeme s vektory o jedenácti složkách,
stačí tedy ze zmíněných tabulek převzít hodnoty z libovolných sloupců
oddílu tabulek, příslušného jedenácti bodům. V tab. 10.1 uvádíme
hodnoty, převzaté z [54, str.~2, n=10].


\begin{table}[!htb]
% \caption{} ... no global caption
\begin{subtable}{0.5\linewidth}
  \centering
  \begin{tabular}{|c|c|c|}
    \hline
    $b_1$ & $b_2$ & $b_3$ \\
    \hline
    -0,476~7313 & -0,458~0286 & ~0,354~7874 \\ %table-data-10.1
    -0,381~3850 & ~0,091~6057 & -0,354~7874 \\ %table-data-10.1
    -0,286~0388 & ~0,335~8876 & -0,354~7874 \\ %table-data-10.1
    -0,190~6925 & ~0,351~1553 & -0,059~1312 \\ %table-data-10.1
    -0,095~3463 & ~0,213~7467 & ~0,236~5249 \\ %table-data-10.1
    ~0,000~0000 & ~0,000~0000 & ~0,354~7874 \\ %table-data-10.1
    ~0,095~3463 & -0,213~7467 & ~0,236~5249 \\ %table-data-10.1
    ~0,190~6925 & -0,351~1553 & -0,059~1312 \\ %table-data-10.1
    ~0,286~0388 & -0,335~8876 & -0,354~7874 \\ %table-data-10.1
    ~0,381~3850 & -0,091~6057 & -0,354~7874 \\ %table-data-10.1
    ~0,476~7313 & ~0,458~0286 & ~0,354~7874 \\ %table-data-10.1
     \hline
  \end{tabular}\\[1.5ex]
  Tab. 10.1
\end{subtable}%
\begin{subtable}{0.5\linewidth}
  \centering
  \begin{tabular}{|c|c|c|}
    \hline
    $a_1$ & $a_2$ & $a_3$ \\
    \hline
    -0,476~731  & -0,481~312  & -0,454~481  \\ %table-data-10.2
    -0,381~385  & -0,380~469  & ~0,088~058  \\ %table-data-10.2
    -0,286~039  & -0,282~680  & ~0,332~340  \\ %table-data-10.2
    -0,190~693  & -0,187~181  & ~0,350~564  \\ %table-data-10.2
    -0,095~346  & -0,093~209  & ~0,216~112  \\ %table-data-10.2
    ~0,000~000  & ~0,000~000  & ~0,003~548  \\ %table-data-10.2
    ~0,095~346  & ~0,093~209  & -0,211~382  \\ %table-data-10.2
    ~0,190~693  & ~0,187~181  & -0,351~747  \\ %table-data-10.2
    ~0,286~039  & ~0,282~680  & -0,339~436  \\ %table-data-10.2
    ~0,381~385  & ~0,380~469  & -0,095~154  \\ %table-data-10.2
    ~0,476~731  & ~0,481~312  & ~0,461~577  \\ %table-data-10.2
     \hline
  \end{tabular}\\[1.5ex]
  Tab. 10.2
\end{subtable}
\end{table}

%\noindent
Vektory $a_i$, vypočtené pomocí takto definovaných vektorů $b_i$
podle vzorci (10.1) a zaokrouhlené na šest desetinných míst, jsou
uspořádány v tab. 10.2.
%

\noindent
Euklidovská  \orig{92} norma skutečné chyby vektoru $a_1$ je potom rovna
$1.10^{-6}$, normy chyb zbývajících vektorů $a_i$ jsou zřejmě řádově
stejné. Budeme proto v dalším klást
%
\begin{align*}
\tag{10.5}
| \delta_a | = 10^{-6}.
\end{align*}
%
Chybami zatížená matice $A$, jejíž prvky jsou sestaveny v tab.
10.2, byla ortogonalizována pomocí procedury ORTON3 (kap. 12)
na počítači ELLIOTT 503 a to jak sekvenční tak selektivní
variantou ortogonalizačního algoritmu. Výsledky ortogonalizace
zachycují tabulky 10.3 a 10.4.

\begin{table}[!htb]
%\caption{}
\begin{subtable}{0.5\linewidth}
  \centering
  \begin{tabular}{|c|c|c|}
    \hline
    $w_1$ & $w_2$ & $w_3$ \\
    \hline
    -0,476~7310  &  -0,458~035  &  ~0,3553  \\
    -0,381~3850  &  ~0,091~614  &  -0,3556  \\
    -0,286~0390  &  ~0,335~888  &  -0,3548  \\
    -0,190~6930  &  ~0,351~181  &  -0,0618  \\
    -0,095~3460  &  ~0,213~687  &  ~0,2424  \\
    ~0,000~0000  &  ~0,000~000  &  ~0,3548  \\
    ~0,095~3460  &  -0,213~687  &  ~0,2306  \\
    ~0,190~6930  &  -0,351~181  &  -0,0565  \\
    ~0,286~0390  &  -0,335~888  &  -0,3547  \\
    ~0,381~3850  &  -0,091~614  &  -0,3539  \\
    ~0,476~7310  &  -0,458~035  &  ~0,3542  \\
     \hline
  \end{tabular}\\[1.5ex]
  Tab. 10.3 ~~Sekvenční varianta
\end{subtable}%
\begin{subtable}{0.5\linewidth}
  \centering
  \begin{tabular}{|c|c|c|}
    \hline
    $w_1$ & $w_2$ & $w_3$ \\
    \hline
    -0,476~7310  &  -0,454~4582  &  -0,35989  \\
    -0,381~3850  &  ~0,088~0532  &  ~0,35651  \\
    -0,286~0390  &  ~0,332~3228  &  ~0,35816  \\
    -0,190~6930  &  ~0,350~5460  &  ~0,06528  \\
    -0,095~3460  &  ~0,216~1009  &  -0,24027  \\
    ~0,000~0000  &  ~0,003~5478  &  -0,35476  \\
    ~0,095~3460  &  -0,211~3712  &  -0,23268  \\
    ~0,190~6930  &  -0,351~7289  &  ~0,05301  \\
    ~0,286~0390  &  -0,339~4185  &  ~0,35136  \\
    ~0,381~3850  &  -0,095~1488  &  ~0,35301  \\
    ~0,476~7310  &  ~0,461~5539  &  -0,34962  \\
     \hline
  \end{tabular}\\[1.5ex]
  Tab. 10.4 ~~Selektivní varianta
\end{subtable}
%
\end{table}






\noindent
V tabulkách \orig{93} 10.5 a 10.6 jsou uvedeny hodnoty euklidovských
norem odpovídajících skutečných chyb a porovnány s odhady vypočtenými
podle vzorců (9.11) a (9.31). Pro větší názornost je přehled
%
doplněn \orig{94} numerickými hodnotami délek $d_i$, z nichž byly
zmíněné odhady počítány. Délky $d_i$, jsou tištěny procedurou ORTON3
pod označením {NORM [i]} (kap. 12).



\begin{table}[ht]
  \centering
  \begin{tabular}{|c|c|c|c|}
    \hline
    $i$ & $d_i$ &
    \begin{tabular}{c}
    Norma skutečné\\chyby $\Vert \delta_{wi} \Vert$ \end{tabular} &
    \begin{tabular}{c}
    Odhad $\Vert \delta_{wi}\Vert$ podle\\(9.11)\end{tabular} \\
    \hline
    1  &  2,1~.~10\TS{0~}  &  1,0~.~10\TS{-6}  &  1,0~.~10\TS{-6} \\
    2  &  2,1~.~10\TS{-2}  &  0,9~.~10\TS{-4}  &  0,5~.~10\TS{-4} \\
    3  &  2,2~.~10\TS{-2}  &  0.8~.~10\TS{-2}  &  0,2~.~10\TS{-2} \\
         \hline
  \end{tabular}\\[1.5ex]
  Tab. 10.5 ~~Normy chyb při užití sekvenční varianty
\end{table}



\begin{table}[htb]
  \centering
  \begin{tabular}{|c|c|c|c|}
    \hline
    $i$ & $d_i$ &
    \begin{tabular}{c}
    Norma skutečné\\chyby $\Vert \delta_{wi} \Vert$y \end{tabular} &
    \begin{tabular}{c}
    Odhad $\Vert \delta_{wi}\Vert$ podle\\(9.11)\end{tabular} \\
    \hline
    1  &  2,1~.~10\TS{0~}  &  1,0~.~10\TS{-6}  &  0,5~.~10\TS{-6} \\
    2  &  2,1~.~10\TS{-2}  &  0,9~.~10\TS{-6}  &  0,9~.~10\TS{-6} \\
    3  &  2,2~.~10\TS{-4}  &  0.9~.~10\TS{-2}  &  1,9~.~10\TS{-2} \\
         \hline
  \end{tabular}\\[1.5ex]
  Tab. 10.6 ~~Normy chyb při užití selektivní varianty
\end{table}


Z tabulek (10.5) a (10.6) je patrno, že v obou případech
odpovídají odhady chyb poměrně přesně skutečnosti.%
\footnote{
   Skutečné chyby jsou u některých vektorů poněkud vyšší než
   \uv{meze} (9.11) a (9.31). Nebyl totiž pro jednoduchost
   respektován rozdíl mezi vstupní normalizací dělením normou
   %
   $\Vert a_i \Vert_\infty$
   %
   užívaným procedurou ORTON3 a dělením normou $\Vert a_i \Vert$,
   předpokládaným při odvození nerovností (9.11) a (9.31) --
   viz též (9.36).
}
%
Ukázali jsme tak zejména, že \Xemph{při užití sekvenční varianty
ortogonalizačního algoritmu může být orakticky dosaženo řádu chyby
stanoveného vzorcem} (9.11). Zkusíme ještě, k jakým závěrům by vedla
aplikace vzorce (9.31) odvozeného pro selektivní variantu na případ
sekvenční; jinými slovy zkusíme posoudit normu chyby
$\Vert \delta_{wi}\Vert$ podle délky $d_i$ určené procedurou
ORTON3. Užijeme-li \orig{95} hodnot z tab. 10.5, bude např. pro $i=3$
%
\begin{align*}
\tag{10.6}
2^2 .\; d_3^{-1} .\; \Vert \delta_a \Vert = 1,8\;.\; 10^{-4}.
\end{align*}
%
Srovnání s hodnotou normy skutečné chyby
%
$\Vert \delta_{w3} \Vert = 0,8 . 10^{-2}$
%
ukazuje, že takový postup může obecně vést k řádovému
podcenění chyby a je proto nepřípustný.


\Subsubsection{B} Druhým příkladem budeme ilustrovat případ zcela odlišných
vlastností, kde řád chyby sekvenční varianty zdaleka nedosáhne
mezí daných vzorcem (9.11). Navíc se ukáže, že na rozdíl od
předcházející úlohy nebude skutečná chyba řádově větší než mez
užívaná u selektivní varianty.

Náš příklad patří mezi úlohy z oboru aproximace funkčních
vztahů, které jak jsme viděli v kap. 4, zpravidla večou ke
špatně podmíněným soustavám normálních rovnic. Konkrétně se budeme
zabývat maticí $A(11 \times 11)$, která vznikne při hledání polynomu
desátého stupně, nabývajícího předem daných hodnot v jedenácti
ekvidistantně rozložených bodech intervalu $<0, 1>$. Snadno se
přesvědčíme, že platí
%
\begin{align*}
\tag{10.7} A =
\left[
\begin{array}{ccccc}
1 & 0            & 0              &  \ldots & 0 \\
1 & 1\;.\;10^{-1} & ~1\;.\;10^{-2}  & \ldots & ~1\;.\;10^{-10} \\
1 & 2\;.\;10^{-1} & 2^2\;.\;10^{-2} & \ldots & 2^{10}.10^{-10} \\
\vdots & \vdots & \vdots &   & \vdots\\
1 & 9\;.\;10^{-1} & 9^2\;.\;10^{-2} & \ldots & 9^{10}.10^{-10} \\
1 & 1            & 1              &  \ldots & 1 \\
\end{array}
\right].
\end{align*}

\noindent
Je možno ukázat, že sloupce $w_i$ $(i=1,2,\ldots,11)$, vzniklé
ortogonalizací odpovídajících sloupců $a_i$ matice $A$ (10.7), jsou
teoreticky tvořeny funkčními hodnotami polynomů ortonormálních na
zmíněné množině bodů. Pro posouzení přesnosti ortogonalizace stačí
tedy porovnat numericky získané vektory $w_i$ s korespondujícími údaji
např. v již citovaných tabulkách [54].

Skutečné chyby posledních šesti vektorů $w_i$ vypočtených
procedurou \orig{96} ORTON3 sekvenční formou ortogonalizačního
algoritmu, charakterizuje tab. 10.7. V tabulce jsou rovněž uvedeny
odhady norem chyb, počítané ze vzorce (9.11), kde bylo v souladu s
velikostí relativní zaokrouhlovací chyby počítače ELLIOTT 503 položeno
$\Vert \delta_a \Vert = 2\,.\,10^{-9}$.\footnote{
Při výpočtu mezí v posledním sloupci tab. 10.7 byly
potřebné součiny $\prod_{j=1}^id_j$ ve vzorci (9.11) aproximovány součiny
experimentálně zjištěných hodnot NORM [i], tištěných počítačem
(kap.12). Při studiu úlohy se ukázalo, že součiny $\prod_{j=1}^id_j$
dobře aproximuje i analytický výraz
%
$\left\{ \det H_i \;.\; \prod_{j=1}^i (2j - 1)\right\}^{1\over2}$,
%
kde $\det H_i$ je determinant segmentu $H_i$ \name{HILBERTOVY} matice,
tvořeného jejími $i$ prvními sloupci a řádky.
}

\begin{table}[htb]
  \centering
  \begin{tabular}{|c|c|c|c|}
    \hline
    $i$ & $d_i$ &
    \begin{tabular}{c}
    Norma skutečné\\chyby $\Vert \delta_{wi} \Vert$y \end{tabular} &
    \begin{tabular}{c}
    Odhad $\Vert \delta_{wi}\Vert$ podle\\(9.11)\end{tabular} \\
    \hline
    6  &  5,0~.~10\TS{-3}  &  9,0~.~10\TS{-7}  &  4,9~.~10\TS{-4}\TS{~*)} \\
    7  &  1,2~.~10\TS{-3}  &  6,0~.~10\TS{-6}  &  4,1~.~10\TS{-1}\TS{~~~} \\
    8  &  2,5~.~10\TS{-4}  &  2,7~.~10\TS{-5}  &    > 1 \\
    9  &  4,6~.~10\TS{-5}  &  1,2~.~10\TS{-4}  &    > 1 \\
   10  &  7,4~.~10\TS{-6}  &  5,9~.~10\TS{-4}  &    > 1 \\
   11  &  8,4~.~10\TS{-7}  &  6,1~.~10\TS{-4}  &    > 1 \\
         \hline
  \end{tabular}\\[1.5ex]
  Tab. 10.7 ~~~ \TS{*)} je $\prod_{j=1}^5 d_j = 1,63 \;.\; 10^{-3}$
\end{table}


\noindent
Z tab. 10.7 je zřejmé, \Xemph{že sekvenční varianta ortogonalizačního
algoritmu dává v daném případě přesnost řádově mnohem vyšší
než udává odhad} (9.11), který zde tak ztrácí jakýkoliv praktický
význam. Skutečná chyba je dokonce menší než mez (9.31), které
bychom užili, kdyby úloha byla vyřešena selektivním algoritmem.
Je např. pro $i=8$:

\begin{align*}
\tag{10.8}
\Vert \delta_{w8} \Vert = 2,7 \;.\; 10^{-5} < 2^7.\;d_8^{-1} =
1,28 \;.\; 10^2 \;.\; 0,4\;.\;10^4 .\; 2\;.\;10^{-9} \approx 10^{-3}.
\end{align*}



Shrneme-li výsledky obou experimentů, docházíme k poznání,
že \Xemph{při užití sekvenční varianty ortogonalizačního algoritmu
mohou nastat dvě extrémní alternativy}:

\begin{enumerate}[label=\alph*)]

\item \Xemph{odhad normy chyby podle vzorce} (9.1) \Xemph{může vést k
vysokému přecenění velikosti chyby} (příklad B),

\item \Xemph{případné užití vzorce} (9.31) \Xemph{může přinést podcenění
velikosti chyby} (příklad A).

\end{enumerate}

Z tohoto \orig{97} hlediska se proto jeví jako účelné užívat pro řešení
špatně podmíněných úloh selektivní varianty ortogonalizačníno
algoritmu, pokud chybí bližší informace o stavbě
ortogonalizované matice $A$.



}%\newcommand{\TS}[1]{\textsuperscript{#1}}
