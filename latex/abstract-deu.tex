\selectlanguage{german}
\addcontentsline{toc}{section}{Zusammenfassung}

\chapter*{Zusammenfassung}
\section*{Lösung von Hauptaufgaben der Ausgllichunssrschnung
          Mittels Orthogonalisierungsmethcde}

\begin{center}
František Charamza
\end{center}

\noindent
%
In der Monographie wird ein numerisch stabile
Orthogonalisierungsalgorithmus (ORTON) für die Lösung linearer
Grundaufgaben der Ausgleichungsrechnung mit unkorrelierten
Beobachtungen abgeleitet und ausführlicherweise analysiert.  Der
Algorithmus kann bei der Ausgleichung vermittelnder und bedingter
Beobachtungen, bei der Ausgleichung vermittelnjer Beobachtungen mit
Bedingungen und bei der Ausgleichung bedingter Beobachtungen mit
Unbekannten zur Berechnung der Verbesserungen, der Unbekannten, der
Funktionen ausgeglichener Grössen und der Gewichtskoeffizienten,
angewendet werden. Der ORTON gehört zu den finiten Algorithmen, die
keine Zusammenstellung der Normalgleichungen erforden.  Neben der
hohen numerischen Stabilität und Universalität des Algorithmus sind
für die Berechnungspraxis seine weiteren Kigenschaften wartvoll,
z.B. diagnostische Eigenschaften, die eine Identifizierung der
Singularität der Aufgabe, Signalisierung eines unentsprechenden
Verlustes gültiger Ziffern u.a., ermöglichen. Der Algorithmus ORTON
kann such zur Losung von Systemen linearer algebraischen Gleichungen,
und zwar nicht nur mit regulären quadratischen Matrizen, sondern auch
mit singulären Matrizen, bzw. rechteckigen Matrizen, angewendet
werden. Es können mittels ihm such homogene Systeme gelöst werden.


Ausser theoretischen Betrachtungen enthält die Arbeit such eine
ausführliche Programmdokumentstion der Prozedur ORTON 3, die den
Algorithmus ORTON in der Sprache ALGOL ausdrückt, und des Programms
ORTON 3 - ZKOUSKA 1, das die Möglichkeit der praktischen Anwendung der
Prozedur ORTON 3 illustriert. Die Prozedur und das Programm können
unmittelbar bei der Berechnung der Ausgleichungsaufgsben auf der
Rechenanlage ELLIOTT 503 und nach kleineren Anordnungen auch auf
anderen, mit einen Übersetzer aus der Sprache ALSOL ausgestatteten
Rechenanlagen angewendet werden.
