
V předcházejících \orig{139} kapitolách jsme se zabývali především
základními úlohami vyrovnávacího počtu a obecnými soustavami
lineárních algebraických rovnic. Pokusili jsme se vypracovat
teorii a najít praktické prostředky pro řešení těchto úloh
univerzálním ortogonalizačním algoritmem. Na závěr ještě v krátkosti
upozorníme na některé další zajímavé aplikační možnosti
ortogonalizační metody. Důkazy náslelujících tvrzení a vzorců nejsou
obtížné a nebudeme je proto uvádět.

\section*{A. Výpočet determinantu}

Pomocí hodnot, získaných při sekvenční ortogonalizaci
libovolné čtvercové regulární matice $A$ $(n \times n)$, lze snadno určit
absolutní hodnotu determinantu této matice. Platí
%
\begin{align*}
  \tag{14.1}
  |\det A| = (\prod_{i=1}^n r_{ii})^{-1}{,}
\end{align*}

\noindent
kde $r_{ii}$ jsou diagonální prvky horní trojúhelníkové matice $R^{-1}$
kterou můžeme najít zobecněnou ortogonalizací jednotkové submatice
např. podle (11.3). Není-li k dispozici matice RU, může být užito
alternativní formule
%
\begin{align*}
  \tag{14.2}
  |\det A| = \prod_{i=1}^n \textrm{ABSMAX} [x,i].\textrm{NORM}[i],
\end{align*}

\noindent
kde $\textrm{ABSMAX}[x,i]$ a $\textrm{NORM}[i]$ jsou hodnoty
tištěné procedurou ORTON3 (kap. 12). Ortogonalizace matice soustavy
rovnic oprav resp. podmínkových rovnic umožňuje analogicky počítat
determinant matice $N$ $(n \times n)$ soustavy normálních rovnic, k
níž vede vyrovnání zprostředkujících nebo nodmínkových pozorování.  Za
předpokladu, že matice $N$ je regulární, platí
%
\begin{align*}
  \tag{14.3}
  \det N = (\prod_{i=1} r_{ii})^{-2}
  = (\textrm{ABSMAX} [x,i].\textrm{NORM}[i])^2,
\end{align*}

\noindent
kde $r_{ii}$ jsou opět diagonální prvky matice vzniklé zobecněnou
ortogonalizací jednotkové submatice - viz např. (3.19) -
a $\textrm{ABSMAX}[x,i]$ a $\textrm{NORM}[i]$ mají stejný význam jako
ve (14.2).

\section*{B. Dodatečné připojení skupiny definičních rovnic}

V praxi dochází někdy k tomu, že soustava definičních rovnic je po
ukončeném vyrovnání doplněna připojením dalších podmínkových rovnic
nebo rovnic oprav. Hledáme potom opravy a příp.  neznámé, vyhovující
takto rozšířené úloze, přičemž jsme vedeni přirozenou snahou využít
výsledků a mezivýsledků, získaných při řešení výchozí úlohy.%
\footnote{Pro jednoduchost uvažujeme v dalším pouze vyrovnání
	  podmínkových A zprostředkujících pozorování.}

Při užití ortogonalizačního algoritmu nečiní zřejmě potíží připojení
podmínkových rovnic při vyrovnání podmínkových pozorování. Toto
konstatování bezprostředně plyne ze zásady postupného zpracování
sloupců ortogonalizované matice. Ze stejného důvodu je rovněž možno
jednoduše připojovat další neznámé při vyrovnání zprostředkujících
pozorování. Stejné závěry platí i pro vypouštění podmínkových rovnic
resp. vypouštění neznámých v rovnicích oprav. Např. z výsledků
vyrovnání zprostředkujících pozorování s $n$ neznámými lze bez
opakovaného vyrovnání poměrně snadno odvodit opravy a neznámé, které
jsou řešením $(n-1)$ úloh vzniklých z výchozí úlohy vypuštěním
postupně posledního, posledních dvou až posledních $(n-1)$ sloupců v
matici rovnic oprav.

Ortogonalizační algoritmus poskytuje prostředky ke schůdnému
řešení i druhé vytčené úlohy, tj. připojení skupiny rovnic oprav
k soustavě rovnic oprav předběžně vyrovnané. Uvažujme
\uv{rozšířenou} soustavu rovnic oprav
%
\begin{align*}
\tag{14.4}
\left[\begin{array}{c} v_1 \\ v_2 \end{array}\right] =
\left[\begin{array}{c} A_1 \\ A_2 \end{array}\right] x +
\left[\begin{array}{c} \ell_1 \\ \ell_2 \end{array}\right] ,
\end{align*}

\noindent
kde $A_1$, $\ell_1$ odpovídají výchozí úloze a $A_2$, $\ell_2$ připojovaným
rovnicím oprav. Nechť $A_1 = W_1 R_1$ je rozklad matice $A_1$,
realizovaný při řešení výchozí úlohy ortogonalizací. Potom bychom mohli
dokázat, že neznámé, určené ve smyslu metody nejmenších čtverců
ze soustavy (14.4) jsou totožné s neznámými, vypočtenými stejnou
metodou z náhradní soustavy rovnic oprav.%
\footnote{Soustavy (14.4) a (14.5) vedou totiž k identickým soustavám
normálních rovnic.}
%
\begin{align*}
\tag{14.5}
\left[\begin{array}{c} w_1 \\ v_2 \end{array}\right] =
\left[\begin{array}{c} R_1 \\ A_2 \end{array}\right] x +
\left[\begin{array}{c} W^T_1\ell_1 \\ \ell_2 \end{array}\right] .
\end{align*}


\noindent
Soustava \orig{141} (14.5) bude mít obecně menší počet rovnic oprav
než soustava (14.4) a její řešení bude proto jednodušší.  Vyrovnáním
(14.5) budou nalezeny neznámé $x$ a opravy $v_2$, odpovídající
(14.4). Opravy $v_1$ lze najít dosazením neznámých do rovnic oprav
(14.4).

Vidíme tedy, že připojení skupiny definičních rovnic
nepřináší při užití ortogonalizačního algoritmu žádné zvláštní
obtíže. Z naznačené možnosti vyrovnání ve dvou etapách vyplývá
potom mj., že ortogonalizace může být v zásadě aplikována při
řešení rozsáhlých souborů rovnic oprav i tehdy, nestačí-li
kapacita operační paměti k současnému uložení všech rovnic oprav.


\section*{C. Souvislost ortogonalizačního algeoritmu s aproximací pomocí
ortogonálních polynomů}

V kap. 4 jsme k aproximaci funkce $f(x)$, definované funkčními
hodnotami na určité množině diskrétních bodů, užili polynomů
(4.15) tvaru
%
\begin{align*}
\tag{14.6}
P_n(x) = \sum_{i=0}^n c_i x^i\Punc{.}
\end{align*}

\noindent
Z literatury je známo [14] , že je často vhodnější pracovat s
polynomy alternativní formy
%
\begin{align*}
\tag{14.7}
P_n(x) = \sum_{i=0}^n d_i \varphi_i(x)\Punc{,}
\end{align*}
%
kde
%
\begin{align*}
\tag{14.8}
\varphi_i(x) = \sum_{j=0}^i b_{ij} x^j
\end{align*}

\noindent
jsou polynomy i-tého stupně $(i=0,1,2,\ldots,n)$ ortonormální na
množině daných bodů $x_1, x_2,\ldots, x_m$ $(m>n)$. Platí proto
%
\begin{align*}
\tag{14.9}
\sum_{k=1}^n \varphi_i(x_k) \varphi_j(x_k) = \delta_{ij} \Punc{.}
\end{align*}


Při \orig{142} hledání aproximačního polynomu (14.7) je třeba v každém
konkrétním případě počítat koeficienty $d_i$, funkční hodnoty
$\varphi_{ik} = \varphi_i(x_k)$, $(k=1,2,\ldots,m)$ a často ještě
koeficienty $b_{ij}$ pro převod polynomu (14.7) na tvar
(14.6). Naznačíme postup pro určení těchto objektů
ortogonalizací. Předpokládejme, že jsme zobecněným ortogonalizačním
algoritmem zpracovali matici $\left[
\begin{array}{c} A \\ E
\end{array}
\right]$
se základní submaticí
%
\begin{align*}
\tag{14.10}
A =
\left[
\begin{array}{cccccccc}
1 & x_1 & x_1^2 & . & . & . & . & x_1^n \\
1 & x_2 & x_2^2 & . & . & . & . & x_2^n \\
. & .   & .     & . & . & . & . & .    \\
. & .   & .     & . & . & . & . & .    \\
1 & x_m & x_m^2 & . & . & . & . & x_m^n \\
\end{array}
\right]
\end{align*}

\noindent
a vedlejší jednotkovou maticí $E$. Získali jsme tak v souladu se
(3.17) a (3.18) matice $W$ a $R^{-1}$, které jsou podle (2.10) spojeny
s maticí $A$ vztahem $W = AR^{-1}$. Označme ještě $f$ vektor, tvořený
danými funkčními hodnotami $f_1,f_2,\ldots,f_m$ funkce $f(x)$. Potom je
možno dokázat,že platí
%
\begin{align*}
        \tag{14.11}
        d = W_Tf\Punc{,}
\end{align*}
%
\begin{align*}
        \tag{14.12}
        b_{ij} = r_{j+1,i+1}\Punc{,} \quad
        (i=0,1,2,\ldots,n)\Punc{,} \quad
        (j=0,1,2,\ldots,i)\Punc{,}
\end{align*}

\noindent
kde vektor $d$ je tvořen hledanými koeficienty v rozvoji (14.7)
a $r_{ij}$ představuje prvek,ležící v j-tém řádku a i-tém sloupci
matice $R^{-1}$. Matice $W$ obsahuje potom hledané funkční hodnoty
polynomů $\varphi_i$ v bodech $x_1, x_2, \ldots, x_,$.
Pro $i=0,1,2,\ldots,n$ a $k=1,2,\ldots.m$ platí
%
\begin{align*}
   \tag{14.13}
   \varphi_{ik} = w_{k,i+1}\Punc{.}
\end{align*}
