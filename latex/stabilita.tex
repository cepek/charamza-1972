
V předcházející kapitole \orig{29} jsme ukázali, že rovnice oprav i
podmínkové rovnice jsou v zásadě řešitelné zobecněným
ortogonalizačním algoritmen, který lze pro jeho zřejmou jednoduchost,
homogenitu a cyklický charakter snadno programovat. Zbývá ještě
posoudit jeho numerickou stabilitu.

Uvažujme nejprve soustavu lineárních algebraických rovnic s
regulární maticí $A$ $(n \times n)$
%
\begin{align*}
  \tag{4.1A}\
  Ax + b = 0
\end{align*}
%
Neznámé $x$ můžeme nalézt bezorostředním řešením této soustavy
(postup A) nebo řešením některé soustavy ekvivalentní (postup
B). Z důvodů, které vysvitnou později, budeme při postupu B
řešit soustavu, vzniklou ze soustavy (4.1A) násobením maticí $A^T$
zleva
\begin{align*}
\tag{4.1B} A^TAx + A^Tb = 0.
\end{align*}
%
Soustavy (4.1A) a (4.1B) jsou ekvivalentní, neboť mají jediné
společné řešení $x$.



Připusťme nyní, že absolutní členy obou soustav jsou zatíženy stejnou
chybou $\delta = \delta b = \delta (A^Tb)$.  Pro zjednodušení našich
úvah omezíme studium stability obou postupů pouze na zkoumání jejich
citlivosti na tuto chybu, tj. budeme se ptát na velikost odpovídající
chyby $\delta x$ neznámých $x$. Ukáže-li se, že u jednoho postupu může
stejné chybě $\delta$ odpovídat v některém smyslu větší chyba $\delta
x$ než u druhého postupu, pak první z nich pokládáme za méně stabilní
a tedy i méně vhodný pro výpočet.




Je známo [15, str.~33], že pro soustavu (4.1A) platí
%
\begin{align*}
  \tag{4.2A} \|\delta x\| / \|x\| &\le \mathrm{cond}(A) \|\delta\|/\|b\|,\\
  \tag{4.3} \mathrm{cond}(A) &= \mu_1 / \mu_n,
\end{align*}
%
kde $\mathrm{cond}(A)$ je číslo podmíněnosti ratice $A$, definované podílem
%
%
maximálního \orig{30} ($\mu_1$) a minimálního ($\mu_n$) singulárního
čísla\footnote{Singulární čísla matice A jsou rovna kladně uvažovaným
odmocninám z charakteristických čísel matice $A^TA$ [15,~str.~15].}
matice A [15, str. 32). Pro soustavu (4.1B) je analogicky
%
\begin{align*}
  \tag{4.2B} \|\delta x\| \ \|x\|
  & \le \mathrm{cond}(A^TA) \|\delta\|/\|A^Tb\|.
\end{align*}
%
Podíly norem v nerovnostech (4.2A) a (4.2B) lze považovat za
míru relativních chyb odpovídajících vektorů. Obě nerovnosti
tak shora omezují relativní chybu vektoru neznámých v závislosti
na relativní chybě vektoru absolutních členů a na čísle
podmíněnosti matice soustavy. Z nerovností (4.2A) a (4.2B) dále
vyplývá, že hledaná citlivost postupů A a B může být měřena
čísly podmíněnosti matic $A$ a $A^TA$. Ukazuje se, že těmito čísly
lze podobně kvantifikovat i závislost chyby neznámých na chybách
prvků matice soustavy a v podstatě i na zaokrounlovacích chybách,
k nimž dochází při vlastním numerickém řešení soustavy [15, str. 36].
Posoudíme nyní vzájemnou velikost obou čísel podmíněnosti.

Lze dokázat, že platí
%
\begin{align*}
  \tag{4.4}   \mathrm{cond} (A^TA) = \left\{\mathrm{cond}(A)\right\}^2 .
\end{align*}
%
Uvážíme-li, že čísla podmíněnosti nemohou být menší než jedna,
pak je zřejmé, že horní mez chyby (4.2B) nemůže být menší než
mez chyby (4.2A). Čísla podmíněnosti, vyskytující se v praxi,
jsou zpravidla řádově větší než jedna. V takových případech je
postup A výrazně stabilnější než postup B. Přestože jde o úlohy
algebraicky ekvivalentní, je při řešení soustavy (4.1B)
realistické očekávat numerické obtíže, které budou tím větší, čím
větší bude číslo podmíněnosti výchozí matice A, nebo jinými slovy,
čím horší bude její podmíněnost.


Pro větší názornost připojíme numerický příklad. Nechť je
např. soustava (4.1A):
%
\begin{align*}
    x_1 &+ 0,9 x_2 - 1,9 = 0\\
0,8 x_1 &+ 0,7 x_2 - 1,5 = 0 ,
\end{align*}
%
soustava \orig{31} (4.1B):
\begin{align*}
  1,64 x_1 &+ 1,46 x_2 - 3,10 = 0\\
  1,46 x_1 &+ 1,30 x_2 - 2,76 = 0 ,
\end{align*}
%
se společným řešením $x_1 = x_2 = 1$. Podrobnější výpočet dává
%
\begin{align*}
  \tag{4.5}   \mathrm{cond}(A) = 1,5\cdot10^2, \qquad
              \mathrm{cond}(A^TA) = 2,2\cdot10^4,
\end{align*}
%
takže podle (4.2A) a (4.2B) je obecně
%
\begin{align*}
\tag{4.6A}   \|\delta x\| / \|x\| &\le 1,5\cdot10^2 \|\delta\|/\|b\| \\
\tag{4.6B}   \|\delta x\| / \|x\| &\le 2,2\cdot10^4 \|\delta\|/\|A^Tb\| .\\
\end{align*}
%
Najdeme numerickou hodnotu relativní chyby řešení, odpovídající
některému konkrétnímu vektoru $\delta$, např.
%
\begin{align*}
\tag{4.7} \delta =(-0,0001 ~~ 0,0001)^T .
\end{align*}
%
Uvedeme opět pouze výsledky výpočtu.\\


\noindent
Soustava (4.1A) :
\begin{align*}
\tag{4.8A}   \delta_x &= (-0,008 ~~ 0,009)^T, &
             \|\delta x\| / \|x\| = 1,5\cdot10^2 \|\delta\|/\|b\|,\hspace{2.5ex}
\end{align*}
soustava (4.1B) :
\begin{align*}
\tag{4.8B}   \delta_x &= (0,690 ~~ -0,775)^T, &
             \|\delta x\| / \|x\| = 2,2\cdot10^4 \|\delta\|/\|A^Tb\| .
\end{align*}


Srovnání skutečně dosažených relativních chyb neznámých se (4.5a)
a (4.6B) ukazuje, že u obou soustav prakticky nastal nejméně
příznivý případ, tj. bylo dosaženo horní meze relativní chyby
neznámých pro danou relativní chybu absolutních členů. Tento jev je
důsledkem záměrné volby vektoru $\delta$, která měla zdůraznit, že ve
vzorcích (4.2A) a (4.2B) může nastat rovnost.

Nerovnostem (4.2A) a (4.2B) je možno dát ještě druhý tvar.
Nechť
\begin{align*}
\tag{4.9} p_a = -log_{10} (\|\delta_a\| / \|a\|),
\end{align*}
%
kde a je libovolný \orig{32} nenulový vektor a $\delta_a$ je nenulový
vektor chyby. Za určitých zjednodušujících předpokladů může
$p_a$ sloužit jako odhad počtu platných cifer složek vektoru
$(a  + \delta_a)$
Logaritmujeme-li např. nerovnost (4.2A), dostaneme
%
\begin{align*}
\tag{4.10} p_x \ge p_b - \log_{10} \mathrm{cond}(A.)
\end{align*}
%
Při řešení soustavy rovnic, jejíž matice má číslo podmíněnosti
%
$\mathrm{cond}(A)$
%
můžeme tedy očekávat, že počet platných cifer
vypočtených neznámých bude až o
%
$\log_{10}\mathrm{cond}(A)$
%
cifer menší než počet s platných cifer absolutních členů. Podle (4.4)
můžeme potom analogicky soudit, že u soustavy s maticí $A^TA$ by mohlo
dojít ke ztrátě až dvojnásobného počtu cifer. V aplikaci na náš
numerický příklad dostaneme
%
\begin{align*}
\tag{4.11}   \log_{10}\mathrm{cond}(A) = 2,2  \qquad
             \log_{10}\mathrm{cond}(A^TA) = 4.3.
\end{align*}
%
Srovnáme-li velikost složek vektoru chyb $\delta_x$ (4.8A)
resp. (4.8B) s teoreticky přesným řešením $x_1 = x_2 = 1$ vidíme, že
počet platných cifer neznámých je skutečně přibližně o 2 resp. 4 cifry
menší než chybou $\delta$ (4.7) determinovaný počet platných cifer
absolutních členů.


Příklad, který jsme právě podrobně probrali ukazuje mj., že numerické
řešení soustavy (4.1B) s maticí $A^TA$ může být podstatně obtížnější
než řešení soustavy (4.1A) s maticí $A$.  V daném kontextu, kdy jsme
uvažovali pouze regulární a tedy čtvercové matice $A$, může být takové
zjištění sice zajímavé, nemá však valného praktického
významu. Nebudeme mít totiž zpravidla důvod k pracnému přechodu ze
soustavy (4.1A) na soustavu (4.1B)%
\footnote{
V některých případech může ovšem být taková transformace
užitečná. Je-li matice $A$ regulární, pak matice $A^TA$ je symetrická
a pozitivně definitní. Tato vlastnost může být někdy cenná.
}


Zcela jiná situace však nastává, je-li matice $A$ obdélníková a
odpovídající soustava je řešena metodou nejmenších čtverců.  Jak jsme
viděli v kap. 3, užívá klasický vyrovnávací postup převodu matice
rovnic oprav $A$ resp. matice podmínkových rovnic $A^T$ první
resp. druhou \name{GAUSSOVOU} transformací [12, str. 35] na čtvercovou
symetrickou matici $N= A^TA$, analogického tvaru jako
%
matice soustavy (4.1B). \orig{33} Matice N je maticí soustavy
normálních rovnic, která se potom vhodnou metodou řeší namísto výchozí
soustavy s maticí $A$ resp. $A^T$.


Číslo podmíněnosti není definováno pro obdélníkové mstice,
nemůžeme tedy porovnat podmíněnost metice $N$ s podmíněností
matice $A$. Přesto však blízká analogie k našemu příkladu nás nutí
očekávat nepříliš dobrou podmíněnost matice $N$ a zároveň nás
vede ke hledání a studiu metod řešících vyrovnávací úlohu přímým
zpracováním matice $A$, tedy bez přechodu k normálním rovnicím.



Na známém příkladě \name{LAUCHLIHO} [26] nejprve ukážeme, že
existují vyrovnávací úlohy, které vedou k libovolně špatně
podmíněným maticím normálních rovnic. \name{LAUCHLI} uvažuje matici rovnic
oprav
%
\begin{align*}
\tag{4.12}   A =
\begin{bmatrix}
1 & 1 & 1 & 1 & 1 \\
\epsilon & 0 & 0 & 0 & 0 \\
0 & \epsilon & 0 & 0 & 0 \\
0 & 0 & \epsilon & 0 & 0 \\
0 & 0 & 0 & \epsilon & 0 \\
0 & 0 & 0 & 0 & \epsilon \\
\end{bmatrix},
\end{align*}
%
kde $\epsilon$ je proměnný parametr. Odpovídající matice normálních
rovnic je
\begin{align*}
\tag{4.13} N = A^TA =
\begin{bmatrix}
1+\epsilon^2 & 1 & 1 & 1 & 1 \\
1 & 1+\epsilon^2 & 1 & 1 & 1 \\
1 & 1 & 1+\epsilon^2 & 1 & 1 \\
1 & 1 & 1 & 1+\epsilon^2 & 1 \\
1 & 1 & 1 & 1 & 1+\epsilon^2 \\
\end{bmatrix}
\end{align*}
%
s číslem podmíněnosti
%
\begin{align*}
\tag{4.14} \mathrm{cond}(N) = (5 + \epsilon^2)\epsilon^{-2},
\end{align*}
%



\noindent
které \orig{34} zřejmě pro $\epsilon \rightarrow 0$ roste nade všechny
meze. Pracujeme-li např. s devítimístnými mantisami v pohyblivé řádové
čárce, potom již hodnota $\epsilon = 10^{-5}$ vede k matici, jejíž
všechny prvky jsou rovny jedné, a která je tedy
singulární. Odpovídající soustava normálních rovnic není potom
jednoznačně řešitelná.


Zbývá ještě porovnat numerickou stabilitu ortogonalizačního algoritmu
se stabilitou klasického vyrovnávacího postupu.  Užijeme opět
ilustrativního příkladu, tentokrát z oblasti aproximace funkčních
vztahů.


Budeme hledat koeficienty $c_i^{(n)}$ polynomů
\begin{align*}
\tag{4.15}
P_n(x)&= \sum_{i=0}^n c_i^{(n)}x^i, \qquad(n=1,2,\ldots,7)
\end{align*}
%
aproximujících ve smyslu metody nejmenších čtverců funkci $f(x)$,
definovanou v $m=11$ diskrétních bodech $x_j$ $(j=1,2,\ldots,11)$
tabulkou
%
\begin{table}[h]
\centering
\begin{tabular}{|c|c|c|c|c|c|c|c|c|c|c|c|}
\hline
$x_j$    & 0   & 0,1 & 0,2 & 0,3 & 0,4 & 0,5 & 0,6 & 0,7 & 0,8 & 0,9 & 1,0\\
\hline
$f(x_j)$ & 0.5 & 0,6 & 0,7 & 0,8 & 0,9 & 1,0 & 1,1 & 1,2 & 1,3 & 1,4 & 1,5\\
\hline
\end{tabular}\\[1.5ex]
Tab. 4.1
%\caption{}
%\label{}
\end{table}
%

\noindent
Z tabulky je patrno, že platí
%
\begin{align*}
\tag{4.16}   f(x_J) = 0,5 + x_j, \qquad (j=1,2,\ldots,11),
\end{align*}
%
takže hledané koeficienty $c_i^{(n)}$ $(n=1,2,...,7)$ jsou teoreticky
rovny
%
\begin{align*}
\tag{4.17}  c_0^{(n)} = 0,5, \qquad
            c_1^{(n)} = 1,   \qquad
            c_i^{(n)} = 0,   \qquad (1<i\le n).
\end{align*}
%





Naši úlohu budeme nyní řešit numericky, přičemž jednou
užijeme přechodu na řešení normálních rovnic a po druhé budeme
ortogonalizovat rovnice oprav\footnote{%
Pro úplnost je třeba poznamenat, že pro řešení uvedené úlohy
byla vyvinuta speciální numericky stabilní metoda, založená
na rekurentním generování polynomů ortogonálních na množině
bodů $x_j$ [14].
}.
%
Zaměříme se na porovnání
přesnosti, kterou obě metody dávají při výpočtu koeficientů
%
$c_i^{(n)}$.
%
Dosaženou přesnost budeme u každého polynomu měřit počtem
platných cifer výsledků, určeným v souladu se (4.9) podle vzorce
%
kde \orig{35} $\|\delta_c\|$ je délka vektoru chyby neznámých a $\|c\|
= 1,12$ je délka vektoru neznámých. Poznamenejme ještě, že vektory
$\delta_c$ jsou v popisované úloze produktem zaokrouhlovacích chyb a
chyb ze zobrazení vstupních dat v počítači.



Hodnoty
%
$\mathrm{P}_{\mathrm{NORM}}$ a
$\mathrm{P}_{\mathrm{ORT}}$
%
odpovídající oběma zkoušeným metodám jsou uvedeny v tab. 4.2. Základem
pro jejich výpočet byly numerické výsledky řešení úlohy na počítači
ELLIOTT 503 (mantisa 29 bitů) podle programu MNČ-E503-VM [20, str.19]
a pomocí procedury ORTC53 (kap. 12).

\begin{table}[h]
\centering
\begin{tabular}{|c|c|c|c|c|}
\hline
\begin{tabular}{c}
 Stupeň\\polynomu\\
\end{tabular} &
$\mathrm{P}_{\mathrm{NORM}}$ &
$\mathrm{P}_{\mathrm{ORT}}$ &
$log_{10}\mathrm{cond}(H)$ &
$log_{10}M^*$\\
\hline
   1  &  7,5  &  8,7  &  1,3  &  0,8 \\  %  1  7,5  8,7  1,3  0,8  scan OCR
   2  &  8,7  &  7,8  &  2,7  &  2,2 \\  %  2  8,7  7,8  2,7  2,2  opraveno!
   3  &  5,2  &  8,0  &  4,2  &  3,6 \\  %  3  5,2  8,0  4,2  3,6
   4  &  3,6  &  6,7  &  5,7  &  5,0 \\  %  4  3,6  6,7  5,7  5,0
   5  &  2,1  &  5,9  &  7,2  &  6,4 \\  %  5  2,1  5,9  7,2  6,4
   6  &  0,7  &  5,2  &  8,7  &  8,0 \\  %  6  0,7  5,2  8,7  8,0
   7  & -0,7  &  4,6  & 10,2  &  9,4 \\  %  7 -0,7  4,6 10,2  9,4
\hline
\end{tabular}\\[1.5ex]
Tab. 4.2
%\caption{}
%\label{}
\end{table}
%



Ve čtvrtém sloupci tabulky jsou uvedeny logaritmy čísel podmíněnosti
tzv. \name{HILBERTOVÝCH} matic $H$ [15, str.99]. Lze dokázat [14], že
matice normálních rovnice $N$, na něž vede první způsob řešení naší
úlohy, jsou blízké \name{HILBERTOVÝM} maticím odpovídajícího
řádu. Budeme proto čísla podmíněnosti matic N aproximovat čísly
podmíněnosti matic $H$. V posledním sloupci tab. 4.2 jsou pro
zajímavost uvedeny logaritmy jiným způsobem definovaných čísel
podmíněnosti $M^*$ matic $N$, která byla zavedena v programovacím
systému MNČ-I [19] a jsou tištěna programem MNČ-E503-VM.

~

\noindent\orig{36} \emph{Analýza údajů v tab. 4.2 nás vede k těmto závěrům:}

\begin{itemize}
\item[1.]
Rozdíly v přesnosti obou uvažovaných metod, aplikovaných k
řešení dobře podmíněných vyrovnávacích úloh%
\footnote{
   Dobře (špatně) podmíněnou vyrovnávací úlohou budeme rozumět
   takovou úlohu,která vede k dobře (špatně) podmíněné matici
   soustavy normálních rovnic.},
%
nejsou podstatné (polynomy prvního a druhého stupně).

\item[2.]
Při řešení úloh špatně podmíněných dává ortogonalizační
metoda řádově vyšší přesnost než metody založené na řešení
soustav normálních rovnic. Existují případy, kdy ortogonalizační
metoda dá použitelné výsledky, zatímco řešení normálních
rovnic selže (polynom sedmého stupně)%
\footnote{
\name{FORSYTHE} [14] komentuje poslední situaci takto:
   \emph{"When $n \ge 7$ or 8, however, one begins to hear strange grumblings
   of discontent in the computing laboratory. The gist of the
   unhappiness is that each method selected to solve the system
   of normal eguations fails somehow for the larger values of n."}
}.

\item[3.]
Při užití obvyklého postupu, vycházejícího z řešení soustavy
normálních rovnic, může dojít ke ztrátě platných cifer rovné až
logaritmu čísla podmíněnosti matice soustavy normálních rovnic.
Ztráta platných cifer při užití metody ortogonalizační není
větší než polovina této hodnoty.

\end{itemize}


Právě formulované závěry jsou v souladu s poznatky, k nimž jsme
dospěli při rozboru řešení soustav (4.1A) a (4.1B) se čtvercovou
maticí a svědčí o přednostech, které má z hlediska přesnosti
ortogonalizační algoritmus před metodou klasickou. Naše zjištění,
která jsme pro názornost vyvodili empiricky z rozboru numerického
příkladu lze, jak ukazuje \name{SCHWARZ} [34, str. 102] odvodit
obecnou úvahou, platnou jak pro vyrovnání zprostředkujících, tak pro
vyrovnání podmínkových pozorování. Pro případ podmínko- vých
pozorování uvádí obecný rozbor \name{DROZDOV} [10], přičemž užívá
norem obdélníkové matice soustavy podmínkových rovnic a matice k ní
pseudoinverzní.
