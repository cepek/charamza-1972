Z počtářské praxe je známo,\orig{3}
že numerické řešení určité
úlohy dvěma různými postupy, založenými na odlišných algebraických
formulacích, obecně vede k různým
výsledkům%
%
{%
\renewcommand{\thefootnote}{\fnsymbol{footnote}}%
\footnote{Text původního vydání byl naskenován, převeden do systému LaTeX a
zveřejněn se souhlasem autora.
Původní číslování stránek
ve vydání z roku 1972 je uváděno na okraji textu.}%
}.
%
Přitom přesnost
výsledků nemusí být v obou případech stejná. Ne všechny postupy
jsou totiž stejně citlivé k chybám ze zaokrouhlování a příp. k
chybám výchozích veličin, ne u všech metod dochází během výpočtu
ke stejně rychlým ztrátám platných cifer. V této souvislosti
je obvyklé srovnávat tzv. numerickou stabilitu různých metod.
Říkáme, že některá metoda je stabilnější než jiná, vede-li k
řešení s více platnými ciframi, tj. méně znehodnocenému zmíněnými
chybami.

Kritické posouzení numerické stability při výběru metody pro
řešení určitého oboru úloh má značný význam, zejména při užití
samočinných počítačů. Zatímco při ručním výpočtu můžeme poměrně
dobře sledovat a případně i regulovat výpočetní přesnost,
automatizuje samočinný počítač dlouhé posloupnosti aritmetických
operací a zbavuje tak počtáře přímé možnosti průběžné analýzy
šíření chyb ve výpočetním procesu. Tato skutečnost je tím
závažnější, že rychlost, vysoká paměťová kapacita a další přednosti
počítačů dávají základní předpoklady pro řešení problémů se
stále rostoucím rozsahem i počtářskou náročností

Numerická stabilita není jedinou vlastností, kterou by měla
mít moderní numerická metoda. Dalšími vítanými vlastnostmi jsou
např. obecnost metody, zaručující řešení co nejširšího okruhu
úloh a jednoduchý, nejlépe cyklický charakter odpovídajícího
algoritmu, vedoucí ke snadnému sestavení programu pro počítač.
Dále je vhodné, aby metoda poskytovala prostředky k automatické
diagnóze úlohy, tj. aby umožnila zavčas identifikovat případné
singularity, signalizovat nepřiměřenou ztrátu platných cífer
apod. Na druhé straně není vzhledem k vysoké rychlosti počítačů
při výběru metody zpravidla\orig{4} rozhodující potřebný počet
aritmetických operací.

V předložené práci odvodíme metodu a algoritmus (ORTON) pro
řešení základních úloh vyrovnávacího počtu, mající uvedené
vlastnosti. Základními úlohami vyrovnávacího počtu budeme
přitom rozumět úlohy následujících čtyř typů
%
\begin{itemize}
\item[a)] vyrovnání zprostředkujících pozorování

\item[b)]vyrovnání podmínkových pozorování

\item[c)] vyrovnání zprostředkujících pozorování s podmínkami
(typ IC - intermediate observations with conditions)

\item[d)] vyrovnání podmínkových pozorování s neznámými - tzv.
obecná úloha vyrovnávacího počtu (typ CU - conditioned
observations with unknowns)
\end{itemize}
%
včetně určení váhových koeficientů. Ve všech čtyřech případech
budeme předpokládat, že odpovídající definiční rovnice (tj.
rovnice oprav resp. podmínkové rovnice) jsou lineární, a že
vyrovnáváme nekorelovaná pozorování. Navržená metoda patří mezi
přímé (finitní) metody řešení vyrovnávacích úloh, nevyžadující
sestavení a řešení normálních rovnic. Opouští tedy tradiční
\name{GAUSSŮV} postup, o němž se v práci ukazuje, že je obecně méně
numericky stabilní. Popisovaná metoda je založena na modifikované
\name{GRAMOVĚ-SCHMIDTOVĚ} ortogonalizaci, korigované doortogonalizací.
Bude vyvozena s užitím principu extremélních vah
\name{H.H.SCHMIDA}.

Vedle problematiky, přímo spojené s odvozením metody a jejím
užitím k řešení základních úloh vyrovnávacího počtu, jsou
v práci studovány ještě další otázky. Např. v kap. 6 je podrobněji
analyzován princip doortogonalizace a jsou formulovány dvě věty
o jeho obecném přínosu ke zpřesnění ortogonalizačního procesu.
Kap. 8 přináší rozbor singularit, které se mohou vyskytnout při
řešení základních vyrovnávacích úloh a naznačuje jeden z možných
způsobů jejich zpracování. V kap. 9 jsou dále definovány dvě
varianty ortogonalizačního algoritmu, tzv. varianta sekvenční a
varianta selektivní, a jsou odvozeny některé odhady pro posouzení
přesnosti výsledků ortogonalizace. Teoretické úvahy z kap. 9
%
%
jsou numericky ilustrovány v kap. 10.\orig{5}
Následující kap. 11
ukazuje, že ortogonalizační metody může být užito i k~řešení
soustav lineárních algebraických rovnic, a to nejen se čtvercovou
regulární maticí, ale i s maticemi singulárními, příp.
obdélníkovými. Lze jí řešit i soustavy homogenní. Algoritmus ORTON
realizuje procedura s označením ORTON3, sestavená v jazyce
ALGOL pro počítač ELLIOTT 503. Vlastní procedura je uložena v
příl. A, její bližší popis obsahuje kap. 12. Možnost praktického
užití procedury ilustruje kap. 13. Závěrečná kapitola
krátce upozorňuje na některé další aplikace ortogonslizačního
algoritmu.

Studium monografie předpokládá u čtenáře vedle znalosti
vyrovnávacího počtu i základní orientaci v dalších matematických
disciplinách, zejména v maticovém počtu, lineární algebře,
numerických metodách a programování. Pro úvodní seznámení
s problematikou těchto oborů, jejichž význam je pro současnou
geodézii nesporný, odkazujeme na následující díla:

\pagebreak[3]

\begin{itemize}

\item[a)] maticový počet
  \begin{itemize}
  \item[] \name{B.BYDŽOVSKÝ}: Úvod do theorie determinentů a matic,
          NČSAV, Praha 1954
  \item[] \name{O.BORŮVKA}: Základy teorie matic. ACADEMIA, Praha 1971
  \end{itemize}

\item[b)] lineární algebra
  \begin{itemize}
  \item[] \name{I.M.GELFAND}: Lineární algebra. NČSAV, Praha 1953
  \item[] \name{G.E.ŠILOV}: Matematičeskij analiz (konečnoměrnyje
    linějnyje prostranstva). Na\-u\-ka, Moskva 1969
  \end{itemize}

\item[c)] numerické metody
  \begin{itemize}
  \item[] \name{O.SLAVÍČEK} a kol.: Základní numerické metody. SNTL,
          Praha 1964
  \item[] \name{D.K.FADDĚJEV}, \name{V.N.FADDĚJEVOVÁ}:
          Numerické metody lineární
          algebry. SNTL, Praha 1964
  \end{itemize}

\item[d)] programování
  \begin{itemize}
  \item[] \name{J.RAICHL}: Programování pro samočinné počítače.
          ACADEMIA, Praha 1972
  \item[] \name{J.RAICHL}: Programování v ALGOLu.
          ACADEMIA, Praha 1971
  \end{itemize}

\item[e)] užití matic ve vyrovnávacím počtu\orig{6}
  \begin{itemize}
  \item[] \name{H.WOLF}: Auscleichungsrechnung nach der Methode der
    kleinsten Ouadrate. Dümmler, Bonn 1968.

  \end{itemize}

\end{itemize}





\noindent Poděkování.\\[-1em]

Autor děkuje recenzentům prof.\ Ing.\ dr.\ J.\ Böhmovi, DrSc.,
prof.\ Ing.\ dr.\ J.\ Vykutilovi, doc.\ RNDr.\ Z.\ Nádeníkovi, DrSc.\ a
Ing.\ V.\ Radouchovi, CSc.\ za podnětné připomínky k rukopisu
práce, které mj.\ přispěly k jejímu zlepšení při konečné úpravě
textu. Děkuje dále svým spolupracovnicím ve Výzkumném
ústavu geodetickém, topografickém a kartografickém,
Aleně Stádníkové za pomoc při výpočtech na počítači a vyhotovení
čistopisu, Janě Vaňkové za doplnění čistopisu vzorci a obrázky.
